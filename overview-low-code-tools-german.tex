% todo: liste mit lowcode frameworks + welche sprache + auf was fokussiert sie sich + was sind hindernisse

\section{Vergleich von Low-Code-Plattformen}
Nachfolgend ist eine Tabelle, die verschiedene Low-Code-Plattformen, ihre Programmiersprachen, Fokus und Hindernisse auflistet.

\begin{longtable}{|m{3cm}|m{3cm}|m{4cm}|m{5cm}|}
\hline
\textbf{Low-Code Plattform} & \textbf{Sprache} & \textbf{Fokus} & \textbf{Hindernisse} \\
\hline
\endhead
Node-RED & Basierend auf Node.js, JavaScript für Funktionen & IoT-Anwendungen, Systemintegration, Workflow-Automatisierung & Skalierbarkeit bei komplexen Systemen, Sicherheitsherausforderungen\\
\hline
OutSystems & Eigene visuelle Sprache, Unterstützung für Java und .NET & Schnelle Anwendungsentwicklung, Mobile Anwendungen, Unternehmensanwendungen & Komplexität bei sehr großen Anwendungen, Potenzielle Anbieterbindung \\
\hline
Mendix & Modellgetriebene Entwicklungsumgebung, Erweiterbar mit JavaScript und Java & Unternehmens-IT-Anwendungen, IoT-Anwendungen, Mobile Anwendungen & Integration mit bestehenden Systemen, Skalierbarkeit bei hochkomplexen Anwendungen \\
\hline
Microsoft Power Apps & Drag-and-Drop-Oberfläche, Excel-ähnliche Ausdrücke, erweiterbar mit JavaScript & Geschäftsanwendungen, Integration mit Microsoft 365 Suite & Begrenzt auf Microsoft-Ökosystem, Weniger Flexibilität für komplexe Anwendungen \\
\hline
Salesforce Lightning Platform & Apex (Java-ähnliche Sprache), SOQL, deklarative Werkzeuge & CRM-bezogene Anwendungen, Geschäftsanwendungen & Lernkurve für proprietäre Sprachen, Höhere Kosten \\
\hline
Appian & Hauptsächlich visuelle Sprache, Unterstützung für benutzerdefinierte Skripte & Prozessautomatisierung, Workflow-Anwendungen, Fallmanagement & Integration mit Nicht-Appian-Systemen, Begrenzte UI-Anpassung \\
\hline
Zoho Creator & Deluge (Skriptsprache von Zoho) & Anwendungen für kleine Unternehmen, Workflow-Automatisierung & Begrenzte Fähigkeiten für komplexe Anwendungen, Abhängigkeit vom Zoho-Ökosystem \\
\hline
\end{longtable}
