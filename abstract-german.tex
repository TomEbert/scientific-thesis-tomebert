Quantencomputing steht an der Schwelle zu einer paradigmatischen
Verschiebung in der Informationsverarbeitung, mit dem Potenzial,
Problemlösungen zu ermöglichen, die über die Grenzen klassischer
Rechenarchitekturen hinausgehen \cite{Shor1999}. Trotz des theoretischen und
praktischen Potenzials des Quantencomputings bleibt der Zugang zu dieser
Technologie aufgrund der inhärenten Komplexität von Quantenalgorithmen
und der erforderlichen tiefgreifenden Kenntnisse in Quantenmechanik und
-informatik limitiert \cite{Chitransh2022}. So unterscheidet sich die Programmierung
von Quantencomputern stark von der im klassischen Bereich \cite{Rieffel2011}.

Im Klassischen bieten Low-Code-Entwicklungsumgebungen einen Ansatz, um
die Barriere für den Einstieg in die Programmierung zu senken. Dies wird
emöglicht durch Abstraktion der technischen Komplexitäten und
Bereitstellung intuitiver, grafischer Entwicklungswerkzeuge \cite{Juhas2022}. Die
Anwendung von Low-Code-Plattformen erweist sich insbesondere bei
einfachen Projekten als vorteilhaft, stößt jedoch bei komplexeren
Anwendungen an ihre Grenzen \cite{Buscher2022}. Diese Arbeit beabsichtigt, zu
untersuchen inwiefern Prinzipien der Low-Code-Entwicklung aus dem
klassischen Computing übertragbar auf Quantencomputing sind. Basierend
auf den Ergebnissen soll eine Schnittstelle zwischen Technologie des
Quantencomputings und der Zugänglichkeit durch Low-Code-Ansätze
geschafft und prototypisch umgesetzt werden. Dabei soll insbesondere
erforscht werden, welche Kriterien und Low-Code-Methoden zur
Modellierung und Implementierung von Quantenschaltkreisen eingesetzt
werden können, um die Anwendung von Quantencomputern einem breiteren
Spektrum von Nutzern zugänglich zu machen. Cerezo et al. zeigen
insbesondere vielversprechende Anwendungen auf wie die Suche nach
Grundzuständen von Molekülen, die Simulation der Dynamik von
Quantensystemen und die Lösung linearer Gleichungssystemen \cite{Cerezo2021}. So
könnten Experten aus den genannten Domänen von Quantencomputern
profitieren, ohne selbst tieferes Fachwissen im Quantencomputing zu
benötigen \cite{Motta2022}.