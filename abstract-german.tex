% Quantencomputing steht an der Schwelle zu einer paradigmatischen
% Verschiebung in der Informationsverarbeitung, mit dem Potenzial,
% Problemlösungen zu ermöglichen, die über die Grenzen klassischer
% Rechenarchitekturen hinausgehen~\cite{Shor1999}. Trotz des theoretischen und
% praktischen Potenzials des Quantencomputings bleibt der Zugang zu dieser
% Technologie aufgrund der inhärenten Komplexität von Quantenalgorithmen
% und der erforderlichen tiefgreifenden Kenntnisse in Quantenmechanik und
% -informatik limitiert~\cite{Chitransh2022}. So unterscheidet sich die Programmierung
% von Quantencomputern stark von der im klassischen Bereich~\cite{Rieffel2011}.

% Im Klassischen bieten Low-Code-Entwicklungsumgebungen einen Ansatz, um
% die Barriere für den Einstieg in die Programmierung zu senken. Dies wird
% emöglicht durch Abstraktion der technischen Komplexitäten und
% Bereitstellung intuitiver, grafischer Entwicklungswerkzeuge~\cite{Juhas2022}. Die
% Anwendung von Low-Code-Plattformen erweist sich insbesondere bei
% einfachen Projekten als vorteilhaft, stößt jedoch bei komplexeren
% Anwendungen an ihre Grenzen~\cite{Buscher2022}. Diese Arbeit beabsichtigt, zu
% untersuchen inwiefern Prinzipien der Low-Code-Entwicklung aus dem
% klassischen Computing übertragbar auf Quantencomputing sind. Basierend
% auf den Ergebnissen soll eine Schnittstelle zwischen Technologie des
% Quantencomputings und der Zugänglichkeit durch Low-Code-Ansätze
% geschafft und prototypisch umgesetzt werden. Dabei soll insbesondere
% erforscht werden, welche Kriterien und Low-Code-Methoden zur
% Modellierung und Implementierung von Quantenschaltkreisen eingesetzt
% werden können, um die Anwendung von Quantencomputern einem breiteren
% Spektrum von Nutzern zugänglich zu machen. Cerezo et al. zeigen
% insbesondere vielversprechende Anwendungen auf wie die Suche nach
% Grundzuständen von Molekülen, die Simulation der Dynamik von
% Quantensystemen und die Lösung linearer Gleichungssystemen~\cite{Cerezo2021}. So
% könnten Experten aus den genannten Domänen von Quantencomputern
% profitieren, ohne selbst tieferes Fachwissen im Quantencomputing zu
% benötigen~\cite{Motta2022}.

Diese Bachelorarbeit widmet sich der Erforschung der Integration von Low-Code-Entwicklungsansätzen in die Quantencomputing-Entwicklung, 
einem Feld, das aufgrund der hohen Komplexität und der erforderlichen spezialisierten Kenntnisse bislang weitgehend Experten vorbehalten war. 
Ziel der Arbeit ist es, zu untersuchen, inwieweit etablierte Low-Code-Plattformen, die in der klassischen Softwareentwicklung Anwendung finden, 
durch gezielte Anpassungen auch für die Entwicklung von Quantencomputing-Anwendungen genutzt werden können.

Die Arbeit stützt sich auf eine systematische Literaturrecherche, die auf den Richtlinien von Kitchenham und Charters basiert. Dabei wurden 
relevante Publikationen identifiziert und analysiert, um die Möglichkeiten und Grenzen von Low-Code-Werkzeugen im Kontext des Quantencomputings 
zu beleuchten. Ein besonderes Augenmerk liegt auf Open-Source-Ansätzen und Model-Driven Engineering (MDE), da diese nicht nur die Barriere für 
den Einstieg in das Quantencomputing senken, sondern auch die Anpassungsfähigkeit und Weiterentwicklung der Plattformen fördern können.

Die Analyse zeigt, dass es zwar bereits erste Ansätze gibt, Low-Code-Plattformen für Quantencomputing nutzbar zu machen, diese jedoch noch in 
einem sehr frühen Stadium sind. Insbesondere fehlen robuste Open-Source-Tools, die die breite Anwendbarkeit und Skalierbarkeit dieser Technologien 
gewährleisten. Die Arbeit identifiziert zudem mehrere Forschungslücken, wie etwa die Notwendigkeit, die Effizienz und Anpassungsfähigkeit von 
Low-Code-Frameworks im Quantencomputing zu verbessern.

Die Ergebnisse dieser Arbeit bieten eine fundierte Grundlage für die Weiterentwicklung von Low-Code-Werkzeugen im Quantencomputing. Zukünftige 
Forschungen sollten sich darauf konzentrieren, diese Werkzeuge zu optimieren und einen Prototyp zu entwickeln, der die Vorteile von Low-Code 
und MDE nutzt, um die Komplexität der Quantenprogrammierung weiter zu reduzieren. Damit könnte ein wichtiger Beitrag zur Demokratisierung des 
Zugangs zu Quantencomputing geleistet werden.

% englisches abstract:
% 
% This bachelor's thesis is dedicated to researching the integration of low-code development approaches in quantum computing development, 
% a field that has so far been largely reserved for experts due to its high complexity and the specialized knowledge required. 
% The aim of the work is to investigate the extent to which established low-code platforms, which are used in classic software development, 
% can also be used for the development of quantum computing applications through targeted adaptations.

% The work is based on a systematic literature review based on the guidelines of Kitchenham and Charters. In the process 
% relevant publications were identified and analyzed in order to shed light on the possibilities and limitations of low-code tools in the context of quantum computing. 
% in the context of quantum computing. Particular attention is paid to open-source approaches and model-driven engineering (MDE), as these not only lower the barrier for 
% barriers to entry into quantum computing, but can also promote the adaptability and further development of platforms.

% The analysis shows that although there are already initial approaches to making low-code platforms usable for quantum computing, these are still at a very early stage. 
% are still at a very early stage. In particular, there is a lack of robust open source tools that ensure the broad applicability and scalability of these technologies. 
% scalability of these technologies. The work also identifies several research gaps, such as the need to improve the efficiency and adaptability of low-code frameworks in quantum computing. 
% low-code frameworks in quantum computing.

% The results of this work provide a sound basis for the further development of low-code tools in quantum computing. Future 
% research should focus on optimizing these tools and developing a prototype that takes advantage of low-code and MDE to reduce the complexity of quantum computing. 
% and MDE to further reduce the complexity of quantum programming. This could make an important contribution to the democratization of 
% access to quantum computing.