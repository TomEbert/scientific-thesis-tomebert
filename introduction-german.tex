% \begin{itemize}
%   \item Einführung in das Thema: Kurze Beschreibung des Themas "Low-Code-Frameworks für Quantencomputing".
%   \item Bedeutung von Quantencomputing: Warum ist Quantencomputing wichtig und welche Potenziale bietet es?
%   \item Herausforderungen im Quantencomputing: Komplexität und Zugangsbarrieren bei der Programmierung von Quantencomputern.
%   \item Rolle von Low-Code-Plattformen: Wie können Low-Code-Plattformen dazu beitragen, die Komplexität zu reduzieren?
%   \item Motivation für die Arbeit: wissenschaftliche Motivation, dieses Thema zu untersuchen.
%   \item Relevanz der Arbeit: Warum ist es wichtig, ein Low-Code-Framework für Quantencomputing zu entwickeln?
% \end{itemize}

Quantencomputing steht an der Schwelle zu einer paradigmatischen
Verschiebung in der Informationsverarbeitung, mit dem Potenzial,
Problemlösungen zu ermöglichen, die über die Grenzen klassischer
Rechenarchitekturen hinausgehen~\cite{Shor1999}. Trotz des theoretischen und
praktischen Potenzials des Quantencomputings bleibt der Zugang zu dieser
Technologie aufgrund der inhärenten Komplexität von Quantenalgorithmen
und der erforderlichen tiefgreifenden Kenntnisse in Quantenmechanik und
-informatik limitiert~\cite{Chitransh2022}. So unterscheidet sich die Programmierung
von Quantencomputern mitunter von der im klassischen Bereich~\cite{Rieffel2011}. 
Cerezo et al.~\cite{Cerezo2021} zeigen insbesondere vielversprechende Anwendungen auf, wie die Suche nach
Grundzuständen von Molekülen, die Simulation der Dynamik von
Quantensystemen und die Lösung linearer Gleichungssystemen. 

In der klassischen Softwareentwicklung bieten Low-Code-Entwicklungsumgebungen einen Ansatz, 
um die Barriere für den Einstieg in die Programmierung zu senken. Dies wird durch die 
Abstraktion technischer Komplexitäten und die Bereitstellung intuitiver, grafischer 
Entwicklungswerkzeuge ermöglicht~\cite{Juhas2022}. Low-Code-Plattformen bieten auch 
erfahrenen Entwicklern leistungsfähige Tools. 
Diese Plattformen ermöglichen durch grafische Layouts eine effizientere Verwaltung komplexer 
Zustände und verbessern somit die Effizienz und Wartbarkeit des Codes. Die
Anwendung von Low-Code-Plattformen erweist sich insbesondere bei
einfacheren Projekten als vorteilhaft, stößt jedoch bei komplexeren
Anwendungen an ihre Grenzen~\cite{Buscher2022}. 
In dieser Arbeit werden die Konzepte der Low-Code-Entwicklung und des Model-Driven Engineering (MDE) 
weitestgehend synonym verwendet. Die tiefergehende Begründung hierfür 
wird im Abschnitt der theoretischen Grundlagen detaillierter beschrieben. 
Beide Ansätze zielen darauf ab, die Softwareentwicklung durch 
den Einsatz von Modellierungstechniken sowohl zu vereinfachen als auch zu beschleunigen und 
unterstützen sogenannte "Citizen Developer", also nicht-professionelle Entwickler, 
indem sie es ermöglichen, dass auch diese aktiver in den Entwicklungsprozess eingebunden werden können.
Während Model-Driven Engineering (MDE) traditionell auf die formale Modellierung und 
Transformationen fokussiert ist, liegt der Schwerpunkt der Low-Code-Entwicklung auf der 
visuellen Programmierung und der Automatisierung vieler Entwicklungsprozesse. 
Diese Unterschiede sind für die Zielsetzung dieser Arbeit von untergeordneter Bedeutung. 
Die gemeinsame Verwendung der Begriffe unterstreicht den integrativen Charakter 
von Low-Code Entwicklung und erleichtert die Diskussion der relevanten Konzepte. 

Auch im Bereich der Quantencomputing-Softwareentwicklung wird der Einsatz von 
Model-Driven Engineering (MDE) bereits erforscht. Gemeinhardt, Garmendia und Wimmer~\cite{gemeinhardt_2021} 
betonen in ihrer Arbeit die Vorteile von MDE-Techniken, wie domänenspezifischen 
Modellierungssprachen und generativen Methoden zur Beschleunigung und Vereinfachung 
der Quantensoftwareentwicklung. Diese Ansätze bieten eine zusätzliche Abstraktionsebene 
über bestehende Quantenhardware und Programmiersprachen, wodurch eine breitere Nutzung 
von Quantencomputing durch Fachleute ermöglicht wird. Diese Perspektive unterstreicht 
die Relevanz der vorliegenden Arbeit, ein Low-Code-Framework für Quantencomputing-Anwendungen 
zu entwickeln, das auf MDE-Prinzipien basiert~\cite{gemeinhardt_2021}.

In dieser Arbeit wird untersucht, inwiefern Prinzipien der Low-Code-Entwicklung aus dem
klassischen Computing übertragbar auf Quantencomputing sind. Basierend auf den Ergebnissen 
soll die Machbarkeit der Low-Code-Entwicklung für Quantencomputing in zukünftigen Arbeiten untersucht werden können. 
Bereits vor der Literaturrecherche werden Kriterien festgelegt, die auf grundlegenden Kenntnissen des Quantencomputings 
basieren. Das Ziel ist es, einen einfacheren Zugang zu Quantencomputern zu ermöglichen, sodass 
Experten aus verschiedenen Domänen profitieren können, ohne tiefere Fachkenntnisse 
im Quantencomputing zu benötigen~\cite{Motta2022}.

In einem Beitrag von Cabot~\cite{Cabot_2020} wird aufgezeigt, dass es im Bereich der 
Low-Code-Entwicklung derzeit an einer starken Open-Source-Community fehlt. 
Open-Source-Modellierungswerkzeuge können helfen, bestehende Lücken zu schließen 
und die damit verbundenen Herausforderungen besser zu adressieren. Während auch 
Closed-Source-Lösungen dies ermöglichen, bieten Open-Source-Ansätze die zusätzliche 
Möglichkeit, die Zugänglichkeit und Anpassungsfähigkeit von Low-Code-Plattformen zu verbessern. 
Dies fördert die Innovationskraft und Zusammenarbeit innerhalb der Community.

Das Hauptziel dieser Arbeit ist die Recherche und Analyse existierender Low-Code-Werkzeuge 
auf ihre Tauglichkeit als Quantencomputing-Tools. Diese Untersuchung bildet die Grundlage 
für die Entwicklung eines Frameworks, das die Prinzipien der Low-Code-Programmierung 
mit den Anforderungen des Quantencomputings verknüpft. Als ergänzendes Werkzeug wird 
eine prototypische Implementierung einer grafischen Sprache für Quantencomputer erstellt. 
Zur Erreichung dieses Ziels werden folgende Aufgaben definiert:

\begin{itemize}
\item
  Eine Recherche und Analyse der relevanten Literatur, um ein
  Verständnis der Low-Code-Konzepte im klassischen Computing zu
  entwickeln, insbesondere in Bezug auf ihre Möglichkeiten und Grenzen.
\item
  Entwicklung der methodischen Grundlagen und des theoretischen Rahmens 
  für ein minimales Low-Code-Quantencomputing-Framework, basierend auf der 
  Analyse übertragbarer Konzepte klassischer Low-Code-Werkzeuge, die zur 
  Erstellung eines Prototyps verwendet werden können.
\end{itemize}


% \textbf{Bemerkung}

% Das Konzept und die Implementierung dieser Arbeit müssen mit der Apache
% v2.0 Lizenz kompatibel sein.

% Der Student muss seine Zeitplanung inklusive einzelner Arbeitsschritte
% und Meilensteinen selbständig verwalten. Ein hilfreicher Leitfaden für
% das Planen und Schreiben wissenschaftlicher Arbeiten kann in~\cite{Deininger200X} und
%~\cite{Zobel2004} gefunden werden. Die bevorzugte Sprache der Arbeit ist Englisch,
% aber Deutsch ist ebenso möglich. Wenn die Arbeit in Deutsch geschrieben
% wird, kann~\cite{Rechenberg2006} für einen guten Schreibstil als Hilfestellung benutzt
% werden.

\newpage
\section*{Aufbau der Arbeit}

Die Arbeit ist in folgender Weise gegliedert:
\paragraph{\cref{chap:theorie} - Theoretischer Hintergund:} Hier werden die theoretischen Grundlagen dieser Arbeit beschrieben, vor 
allem Low-Code-Entwicklung, Model-Driven Engineering und Quantencomputing.
% \paragraph{\cref{chap:related} - Verwandte Arbeiten:} Stellt relevante wissenschaftliche Arbeiten vor, die sich mit ähnlichen Themen befassen.
\paragraph{\cref{chap:slr} - Systematische Literaturrecherche:} Beschreibt die Methodik der systematischen Literaturrecherche sowie die Ergebnisse 
und Interpretation dieser.
% \paragraph{\cref{chap:prototyp} - Prototypische Entwicklung des Low-Code-Frameworks für Quantencomputing:} Beschreibt die 
% Entwicklung des Prototyps, einschließlich der Anforderungen, Architektur, Implementierung und Herausforderungen.
\paragraph{\cref{chap:zusfas} - Fazit und Ausblick:} Hier werden die Ergebnisse der Arbeit zusammengefasst und Anknüpfungspunkte vorgestellt.
