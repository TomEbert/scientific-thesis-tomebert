\begin{itemize}
  \item Einführung in das Thema: Kurze Beschreibung des Themas "Low-Code-Frameworks für Quantencomputing".
  \item Bedeutung von Quantencomputing: Warum ist Quantencomputing wichtig und welche Potenziale bietet es?
  \item Herausforderungen im Quantencomputing: Komplexität und Zugangsbarrieren bei der Programmierung von Quantencomputern.
  \item Rolle von Low-Code-Plattformen: Wie können Low-Code-Plattformen dazu beitragen, die Komplexität zu reduzieren?
  \item Motivation für die Arbeit: wissenschaftliche Motivation, dieses Thema zu untersuchen.
  \item Relevanz der Arbeit: Warum ist es wichtig, ein Low-Code-Framework für Quantencomputing zu entwickeln?
\end{itemize}

Quantencomputing steht an der Schwelle zu einer paradigmatischen
Verschiebung in der Informationsverarbeitung, mit dem Potenzial,
Problemlösungen zu ermöglichen, die über die Grenzen klassischer
Rechenarchitekturen hinausgehen \cite{Shor1999}. Trotz des theoretischen und
praktischen Potenzials des Quantencomputings bleibt der Zugang zu dieser
Technologie aufgrund der inhärenten Komplexität von Quantenalgorithmen
und der erforderlichen tiefgreifenden Kenntnisse in Quantenmechanik und
-informatik limitiert \cite{Chitransh2022}. So unterscheidet sich die Programmierung
von Quantencomputern stark von der im klassischen Bereich \cite{Rieffel2011}.

In der klassischen Softwareentwicklung bieten Low-Code-Entwicklungsumgebungen einen Ansatz, um
die Barriere für den Einstieg in die Programmierung zu senken. Dies wird
emöglicht durch Abstraktion der technischen Komplexitäten und
Bereitstellung intuitiver, grafischer Entwicklungswerkzeuge \cite{Juhas2022}. Die
Anwendung von Low-Code-Plattformen erweist sich insbesondere bei
einfachen Projekten als vorteilhaft, stößt jedoch bei komplexeren
Anwendungen an ihre Grenzen \cite{Buscher2022}. Diese Arbeit beabsichtigt, zu
untersuchen inwiefern Prinzipien der Low-Code-Entwicklung aus dem
klassischen Computing übertragbar auf Quantencomputing sind. Basierend
auf den Ergebnissen soll eine Schnittstelle zwischen Technologie des
Quantencomputings und der Zugänglichkeit durch Low-Code-Ansätze
geschafft und prototypisch umgesetzt werden. Dabei soll insbesondere
erforscht werden, welche Kriterien und Low-Code-Methoden zur
Modellierung und Implementierung von Quantenschaltkreisen eingesetzt
werden können, um die Anwendung von Quantencomputern einem breiteren
Spektrum von Nutzern zugänglich zu machen. Cerezo et al. zeigen
insbesondere vielversprechende Anwendungen auf wie die Suche nach
Grundzuständen von Molekülen, die Simulation der Dynamik von
Quantensystemen und die Lösung linearer Gleichungssystemen \cite{Cerezo2021}. So
könnten Experten aus den genannten Domänen von Quantencomputern
profitieren, ohne selbst tieferes Fachwissen im Quantencomputing zu
benötigen \cite{Motta2022}.


Das Hauptziel dieser Arbeit ist die Entwicklung eines Frameworks, das
die Prinzipien der Low-Code-Programmierung mit den Anforderungen des
Quantencomputings verknüpft. Diese Verbindung soll durch die
prototypische Implementierung einer grafischen Sprache für
Quantencomputer realisiert werden. Zur Erreichung dieses Ziels werden
folgende Aufgaben definiert:

\begin{itemize}
\item
  Eine Recherche und Analyse der relevanten Literatur, um ein
  Verständnis der Low-Code-Konzepte im klassischen Computing zu
  entwickeln, insbesondere in Bezug auf ihre Möglichkeiten und Grenzen.
\item
  Prototypisches Design und Implementierung einer minimalen Version
  eines Low-Code-Quantencomputing Frameworks unter Verwendung von
  übertragbaren Konzepten von Low-Code Tools im klassischen Computing
  aus der vorausgegangen Analyse.
\end{itemize}

Durch die Umsetzung dieser Aufgaben wird ein neues prototypisches
Framework entwickelt, das die Anwendung von Quantencomputing durch
Low-Code-Methoden zugänglicher und benutzerfreundlicher macht, und somit
die Entwicklung und das Testen von Quantum Circuits vereinfacht.

% \textbf{Bemerkung}

% Das Konzept und die Implementierung dieser Arbeit müssen mit der Apache
% v2.0 Lizenz kompatibel sein.

% Der Student muss seine Zeitplanung inklusive einzelner Arbeitsschritte
% und Meilensteinen selbständig verwalten. Ein hilfreicher Leitfaden für
% das Planen und Schreiben wissenschaftlicher Arbeiten kann in \cite{Deininger200X} und
% \cite{Zobel2004} gefunden werden. Die bevorzugte Sprache der Arbeit ist Englisch,
% aber Deutsch ist ebenso möglich. Wenn die Arbeit in Deutsch geschrieben
% wird, kann \cite{Rechenberg2006} für einen guten Schreibstil als Hilfestellung benutzt
% werden.

\section*{Aufbau der Arbeit}

Die Arbeit ist in folgender Weise gegliedert:
\paragraph{\cref{chap:theorie} - Theoretischer Hintergund:} Hier werden die theoretischen Grundlagen dieser Arbeit beschrieben, vor 
allem Low-Code-Entwicklung, Model-Driven Engineering und Quantencomputing.
\paragraph{\cref{chap:slr} - Systematische Literaturrecherche:} beschreibt die Methodik der systematischen Literaturrecherche sowie die Ergebnisse 
und Interpretation dieser.
\paragraph{\cref{chap:prototyp} - Prototypische Entwicklung des Low-Code-Frameworks für Quantencomputing:} beschreibt die prototypische 
Entwicklung des Low-Code-Frameworks für Quantencomputing, einschließlich der Anforderungen, Architektur, Implementierung und Herausforderungen
\paragraph{\cref{chap:zusfas} - Fazit und Ausblick:} Hier werden die Ergebnisse der Arbeit zusammengefasst und Anknüpfungspunkte vorgestellt.


\begin{itemize}
  \item Kapitel 1: Einleitung
      \begin{itemize}
          \item Hintergrund und Motivation
          \item Zielsetzung der Arbeit
          \item Forschungsfrage
          \item Aufbau der Arbeit
      \end{itemize}
  \item Kapitel 2: Theoretischer Hintergrund
      \begin{itemize}
          \item Grundlagen von Low-Code und Model-Driven Engineering (MDE)
          \item Grundlagen des Quantencomputings
      \end{itemize}
  \item Kapitel 3: Systematische Literaturrecherche
      \begin{itemize}
          \item Methodik der Literaturrecherche
          \item Ergebnisse der Literaturrecherche
          \item Interpretation der Literaturrecherche
      \end{itemize}
  \item Kapitel 4: Prototypische Entwicklung des Low-Code-Frameworks für Quantencomputing
      \begin{itemize}
          \item Anforderungen und Ziele
          \item Architektur des Frameworks
          \item Integration von Model-Driven Engineering
          \item Implementierung der grafischen Sprache
          \item Limitationen und Herausforderungen
      \end{itemize}
  \item Kapitel 5: Fazit und Ausblick
      \begin{itemize}
          \item Zusammenfassung der Ergebnisse
          \item Implikationen für die Praxis
          \item Zukünftige Forschungsperspektiven
      \end{itemize}
\end{itemize}