In diesem Abschnitt werden die theoretischen Grundlagen erläutert, die für das Verständnis des Forschungsgebiets 
erforderlich sind. Dazu gehören die Konzepte von Low-Code-Entwicklung, Model-Driven Engineering (MDE) sowie 
Quantencomputing. Diese Konzepte bilden die Grundlage für die Entwicklung eines Low-Code-Frameworks für Quantencomputing, 
das hypothetisch die Anwendung von Quantencomputing durch Low-Code-Methoden zugänglicher und benutzerfreundlicher machen könnte.

Low-Code-Entwicklung ermöglicht es, Anwendungen mit minimalem Programmieraufwand zu erstellen, was die 
Entwicklungszeit verkürzt und die Zugänglichkeit für Citizen-Programmer erhöht \cite{Bock_2021}. Model-Driven Engineering (MDE) 
nutzt formale Modelle, um den Entwicklungsprozess zu automatisieren und zu standardisiere \cite{gemeinhardt_2021}. Beide Ansätze verfolgen 
das Ziel, die Komplexität der Softwareentwicklung zu reduzieren. 

Die Anwendung dieser Methoden im Quantencomputing ist Gegenstand dieser Arbeit. Es wird 
untersucht, ob und wie die Prinzipien der Low-Code-Entwicklung und des MDE auf die 
Entwicklung von Quantencomputing-Anwendungen übertragen werden können. Ziel ist es, 
zu erforschen, ob diese Ansätze die Entwicklung von Quantenalgorithmen und -anwendungen effizienter 
gestalten und einer breiteren Nutzergruppe zugänglich machen können. Dies könnte eine entscheidende 
Rolle bei der Weiterentwicklung der Quantencomputing-Technologie spielen \cite{Bock_2021} \cite{gemeinhardt_2021}.

\section{Low-Code-Entwicklung und Model-Driven Engineering}
% \begin{itemize}
%     \item Definition und Prinzipien
%         \begin{itemize}
%             \item Beschreibung der Low-Code-Entwicklung
%                 \begin{itemize}
%                     \item Visuelle Entwicklungsumgebungen
%                     \item Minimierung manuellen Codierens
%                 \end{itemize}
%             \item Beschreibung des Model-Driven Engineering (MDE)
%                 \begin{itemize}
%                     \item Nutzung formaler Modelle zur Softwareentwicklung
%                     \item Transformation von Modellen in Code
%                 \end{itemize}
%             \item Gemeinsame Prinzipien
%                 \begin{itemize}
%                     \item Abstraktion und Automatisierung
%                     \item Reduzierung der Komplexität
%                     \item Steigerung der Effizienz und Produktivität
%                 \end{itemize}
%         \end{itemize}
%     \item Synonyme Verwendung der Begriffe
%         \begin{itemize}
%             \item Erklärung der synonymen Verwendung in dieser Arbeit
%             \item Begründung: Gemeinsame Ziele und Methoden
%             \item Vorteile der integrativen Betrachtung
%         \end{itemize}
%     \item Unterschiede zwischen Low-Code und MDE
%         \begin{itemize}
%             \item Fokussierung der Low-Code-Entwicklung
%                 \begin{itemize}
%                     \item Zielgruppe: "Citizen Developers"
%                     \item Visuelle Programmierung und Benutzerfreundlichkeit
%                 \end{itemize}
%             \item Fokussierung von MDE
%                 \begin{itemize}
%                     \item Formale Modellierung und Transformationen
%                     \item Nutzung in komplexen, technischen Projekten
%                 \end{itemize}
%             \item Relevanz der Unterschiede für diese Arbeit
%                 \begin{itemize}
%                     \item Unterschiede werden kurz behandelt
%                     \item Hauptfokus auf gemeinsamen Prinzipien und Nutzen
%                 \end{itemize}
%         \end{itemize}
% \end{itemize}

\paragraph{Low-Code-Entwicklung}
Low-Code-Entwicklung ist ein Ansatz, der darauf abzielt, die Entwicklung von Anwendungen
durch die Automatisierung von Prozessen und die Reduzierung der Notwendigkeit von
Handcodierung zu vereinfachen. Dieser Ansatz ermöglicht es Fachexperten, Anwendungen
zu erstellen, ohne umfassende Programmierkenntnisse zu besitzen. Low-Code-Plattformen
bieten eine Vielzahl von Funktionen, die es ermöglichen, Anwendungen zu erstellen, zu
testen und zu implementieren. Dazu gehören unter anderem visuelle Entwicklungsumgebungen,
die es ermöglichen, Anwendungen durch das Ziehen und Ablegen von Komponenten zu erstellen,
sowie die Integration von Datenbanken und anderen Systemen. Low-Code-Plattformen bieten
auch Funktionen zur Automatisierung von Prozessen, die es ermöglichen, Anwendungen zu
erstellen, die auf Ereignisse reagieren und sich an veränderte Bedingungen anpassen können.

Ein zentrales Merkmal von Low-Code-Plattformen sind visuelle Entwicklungsumgebungen. Diese Umgebungen ermöglichen 
es, Anwendungen durch das Ziehen und Ablegen von Komponenten zu erstellen \cite{Bock_2021}. Dadurch wird 
der Entwicklungsprozess nicht nur beschleunigt, sondern auch intuitiver gestaltet, sodass auch Nutzer ohne 
tiefgehende Programmierkenntnisse in der Lage sind, komplexe Anwendungen zu entwickeln. Shridhar \cite{Shridhar_2021} betont, 
dass diese visuellen Entwicklungswerkzeuge die Barriere für die Softwareentwicklung erheblich senken und die 
Produktivität erhöhen, indem sie die Notwendigkeit von komplexen Programmiersprachen minimieren.

Ein weiteres wichtiges Element ist die Minimierung manuellen Codierens. Low-Code-Plattformen bieten Funktionen, 
die es ermöglichen, Anwendungen weitgehend ohne traditionellen Code zu entwickeln. Stattdessen werden Prozesse 
und Logiken durch visuelle Modelle und Konfigurationen abgebildet \cite{Bock_2021}. Dies reduziert nicht nur 
die Fehleranfälligkeit, sondern erleichtert auch die Wartung und Anpassung von Anwendungen. Alamin und Iqbal \cite{Alamin_2021} 
führen aus, dass diese Ansätze Entwickler dabei unterstützen, sich auf die Geschäftslogik und die spezifischen Anforderungen 
der Anwendung zu konzentrieren, anstatt sich mit den Feinheiten der Programmierung auseinanderzusetzen.

Zusätzlich bieten Low-Code-Plattformen Funktionen zur Automatisierung von Prozessen. Diese ermöglichen es, Anwendungen 
zu erstellen, die auf Ereignisse reagieren und sich an veränderte Bedingungen anpassen können. Durch die Integration 
von Datenbanken und anderen Systemen wird zudem die Erstellung umfangreicher, datengetriebener Anwendungen 
vereinfacht \cite{Bock_2021}. Shridhar \cite{Shridhar_2021} und Alamin und Iqbal \cite{Alamin_2021} zeigen auf, dass diese 
Automatisierungsfunktionen nicht nur die Effizienz steigern, sondern auch die Skalierbarkeit und Flexibilität 
der entwickelten Anwendungen erhöhen.

Ein weiterer Vorteil der Low-Code-Entwicklung ist die Förderung der Zusammenarbeit zwischen verschiedenen 
Stakeholdern. Die visuelle Natur der Entwicklungsumgebungen ermöglicht es, dass Geschäftsanalysten, Entwickler 
und Endnutzer enger zusammenarbeiten können, was zu einer besseren Übereinstimmung der entwickelten Software mit 
den geschäftlichen Anforderungen führt \cite{Shridhar_2021}. Alamin und Iqbal \cite{Alamin_2021} untersuchen in ihrer empirischen 
Studie die Herausforderungen, denen sich Entwickler in Low-Code-Umgebungen gegenübersehen, und zeigen auf, dass trotz 
der Vorteile auch Aspekte wie Plattformabhängigkeit und eingeschränkte Anpassungsfähigkeit bedacht werden müssen.

Insgesamt bietet die Low-Code-Entwicklung viele Vorteile, darunter die beschleunigte Entwicklung, die Reduzierung 
der Handcodierung und die Verbesserung der Zusammenarbeit. Diese Vorteile machen Low-Code-Plattformen zu einer 
attraktiven Option für die Entwicklung moderner Anwendungen, insbesondere im Kontext von Quantencomputing, wo 
komplexe Algorithmen und Datenverarbeitung eine zentrale Rolle spielen könnten.

Die wesentlichen Charakteristika der Low-Code-Entwicklung sowie deren Bewertung als Chance oder Risiko, sind in 
Tabelle \ref{tab:lowcode_characteristics} zusammengefasst.

\begin{table}[h!]
    \centering
    \begin{tabular}{|p{4cm}|p{6cm}|p{4cm}|}
    \hline
    \textbf{Charakteristik} & \textbf{Beschreibung} & \textbf{Chance/Risiko} \\
    \hline
    Visuelle Entwicklungsumgebungen & Anwendungen werden durch Ziehen und Ablegen von Komponenten erstellt. & Chance: Erhöht die Zugänglichkeit und Benutzerfreundlichkeit \\
    \hline
    Minimierung manuellen Codierens & Prozesse und Logiken werden durch visuelle Modelle und Konfigurationen abgebildet, ohne traditionellen Code. & Chance: Reduziert Fehleranfälligkeit, Risiko: Eingeschränkte Flexibilität \\
    \hline
    Automatisierung von Prozessen & Anwendungen reagieren auf Ereignisse und passen sich an veränderte Bedingungen an. & Chance: Steigert Effizienz und Anpassungsfähigkeit \\
    \hline
    Integration von Datenbanken und Systemen & Einfachere Erstellung datengetriebener Anwendungen durch Integration mit Datenbanken und anderen Systemen. & Chance: Erhöht die Funktionalität und Datenzugänglichkeit \\
    \hline
    Förderung der Zusammenarbeit & Erleichtert die Zusammenarbeit zwischen Geschäftsanalysten, Entwicklern und Endnutzern. & Chance: Verbessert die Übereinstimmung der Software mit den Geschäftsanforderungen \\
    \hline
    Schnellere Entwicklungszeiten & Verkürzt die Entwicklungszeiten durch reduzierte Programmieranforderungen. & Chance: Erhöht die Produktivität \\
    \hline
    Zugänglichkeit für Nicht-Programmierer & Ermöglicht auch Fachexperten ohne tiefgehende Programmierkenntnisse die Entwicklung von Anwendungen. & Chance: Erweitert den Entwicklerkreis \\
    \hline
    Flexibilität und Skalierbarkeit & Erhöht die Flexibilität und Skalierbarkeit der entwickelten Anwendungen. & Chance: Erleichtert Anpassungen und Wachstum \\
    \hline
    Plattformabhängigkeit & Potenzielle Abhängigkeit von spezifischen Low-Code-Plattformen und deren Limitierungen. & Risiko: Erhöht die Abhängigkeit und mögliche Einschränkungen \\
    \hline
    Wartung und Anpassung & Erleichtert die Wartung und Anpassung von Anwendungen durch visuelle Modelle. & Chance: Verbessert die Wartbarkeit und Anpassungsfähigkeit \\
    \hline
    \end{tabular}
    \caption{Charakteristika der Low-Code-Entwicklung und deren Bewertung als Chance oder Risiko \cite{Bock_2021, Shridhar_2021, Alamin_2021}}
    \label{tab:lowcode_characteristics}
\end{table}
    
\paragraph{Model-Driven Engineering}
Model-Driven Engineering (MDE) ist ein Ansatz zur Softwareentwicklung, der darauf abzielt, die Entwicklung von 
Anwendungen durch die Verwendung von Modellen zu vereinfachen. MDE ermöglicht es Entwicklern, Anwendungen zu erstellen, 
indem sie Modelle verwenden, die eine abstrakte Darstellung der Anwendung und ihrer Funktionalität darstellen. Diese 
Modelle können dann verwendet werden, um den Code für die Anwendung automatisch zu generieren. MDE bietet eine Reihe 
von Vorteilen, darunter die Möglichkeit, Anwendungen schneller zu entwickeln, die Wiederverwendung von Modellen und 
die Möglichkeit, die Qualität der Anwendungen zu verbessern.

Ein wesentliches Merkmal von MDE ist die Nutzung formaler Modelle zur Softwareentwicklung. Diese Modelle bieten 
eine präzise und verständliche Beschreibung der Struktur und des Verhaltens der zu entwickelnden 
Software \cite{Schmidt_2006}. Durch die Abstraktion von technischen Details ermöglichen formale Modelle 
eine klare Kommunikation zwischen den verschiedenen Stakeholdern im Entwicklungsprozess und tragen dazu bei, 
Missverständnisse zu vermeiden. Formale Modelle können in verschiedenen Formen vorliegen, darunter UML-Diagramme, 
ER-Diagramme und BPMN-Modelle, die jeweils spezifische Aspekte der Software und ihrer Prozesse abbilden \cite{Selic_2003}.

Ein weiteres zentrales Element von MDE ist die Transformation von Modellen in Code. Dieser Prozess, der als 
Model-to-Code-Transformation bezeichnet wird, ermöglicht es, aus den erstellten Modellen automatisch lauffähigen 
Code zu generieren \cite{brambilla2017model}. Dies führt nicht nur zu einer Beschleunigung der Entwicklung, sondern 
reduziert auch die Fehleranfälligkeit, da manuelle Codierarbeiten minimiert werden. Durch die Automatisierung der 
Code-Generierung wird zudem die Konsistenz zwischen den Modellen und dem finalen Code sichergestellt. Verschiedene 
Werkzeuge und Plattformen wie Eclipse Modeling Framework (EMF) und Acceleo unterstützen diesen Transformationsprozess 
und ermöglichen eine nahtlose Integration in den Entwicklungsworkflow \cite{steinberg2008emf}.

MDE fördert außerdem die Wiederverwendbarkeit von Modellen und Komponenten. Durch die Definition wiederverwendbarer 
Modellbausteine können Entwickler effizienter arbeiten und die Qualität der Anwendungen erhöhen. Dies ermöglicht es, 
bewährte Lösungen und Muster zu nutzen und so den Entwicklungsaufwand zu reduzieren \cite{France_2007}. Zusätzlich 
trägt MDE zur Verbesserung der Wartbarkeit und Erweiterbarkeit von Software bei, da Änderungen am Modell automatisch 
im generierten Code reflektiert werden können.

Trotz der vielen Vorteile gibt es auch Risiken und Herausforderungen, die bei der Anwendung von MDE beachtet werden 
müssen. Zu den Risiken gehört die potenzielle Komplexität der Modelle, die schwer zu verstehen und zu pflegen sein 
können. Darüber hinaus besteht eine Abhängigkeit von speziellen Modellierungs- und Code-Generierungswerkzeugen, was 
die Flexibilität einschränken kann. Die Qualität des automatisch generierten Codes kann variieren und zu 
Performance-Problemen und Fehlern führen. Zudem erfordert MDE spezifische Schulungen und Wissen, was zusätzliche
Kosten und Aufwand bedeutet. Herausforderungen bestehen auch in der Skalierbarkeit und Integration von MDE-Lösungen 
in bestehende Systeme sowie in der Modell-zu-Code-Abstraktionslücke, die dazu führen kann, dass nicht alle Details 
und Optimierungen im generierten Code abgebildet werden.

Zusammenfassend lässt sich sagen, dass MDE eine umfassende Methodik zur Softwareentwicklung bietet, die durch 
die Nutzung formaler Modelle und die Automatisierung der Code-Generierung sowohl die Effizienz als auch die 
Qualität der entwickelten Anwendungen steigern kann. Die wesentlichen Aspekte und Vorteile von MDE sind 
in Tabelle \ref{tab:mde_characteristics} zusammengefasst. Diese Übersicht bietet eine klare Struktur zur 
Einordnung der MDE-Methoden, indem sie die verschiedenen Merkmale, deren Beschreibungen und die daraus 
resultierenden Vorteile und Risiken aufzeigt.

\begin{table}[h!]
    \centering
    \begin{tabular}{|p{4cm}|p{6cm}|p{4cm}|}
    \hline
    \textbf{Charakteristik} & \textbf{Beschreibung} & \textbf{Chance/Risiko} \\
    \hline
    Nutzung formaler Modelle & Verwendung von präzisen und verständlichen Modellen zur Beschreibung der Struktur und des Verhaltens der Software. & Chance: Erhöht die Klarheit und Kommunikation zwischen Stakeholdern \\
    \hline
    Transformation von Modellen in Code & Automatische Generierung von lauffähigem Code aus den erstellten Modellen. & Chance: Beschleunigt die Entwicklung und reduziert Fehleranfälligkeit \\
    \hline
    Wiederverwendbarkeit von Modellen & Definition wiederverwendbarer Modellbausteine, um Effizienz und Qualität zu steigern. & Chance: Reduziert Entwicklungsaufwand und erhöht Qualität \\
    \hline
    Verbesserung der Wartbarkeit & Änderungen am Modell werden automatisch im generierten Code reflektiert. & Chance: Erhöht Wartbarkeit und Erweiterbarkeit \\
    \hline
    Komplexität der Modelle & Erstellung sehr komplexer Modelle, die schwer zu verstehen und zu pflegen sind. & Risiko: Erhöht die Komplexität und Wartungsaufwand \\
    \hline
    Werkzeugabhängigkeit & Abhängigkeit von speziellen Modellierungs- und Code-Generierungswerkzeugen. & Risiko: Einschränkung der Flexibilität durch Abhängigkeit von Anbietern \\
    \hline
    Qualität der generierten Codebasis & Variierende Qualität des automatisch generierten Codes. & Risiko: Kann zu Performance-Problemen und Fehlern führen \\
    \hline
    Kosten und Aufwand für Schulungen & Notwendigkeit spezifischer Schulungen und Wissen für die Nutzung von MDE. & Risiko: Erhöht die Kosten und den Aufwand für Schulungen \\
    \hline
    Skalierbarkeit und Integration & Herausforderungen bei der Integration und Skalierung von MDE-Lösungen. & Risiko: Komplexität bei groß angelegten Projekten \\
    \hline
    Modell-zu-Code-Abstraktionslücke & Gefahr, dass Modelle nicht alle Details und Nuancen des Codes abbilden. & Risiko: Fehlen von Details und Optimierungen im generierten Code \\
    \hline
    \end{tabular}
    \caption{Charakteristika des Model-Driven Engineering (MDE), deren Chancen und Risiken \cite{Schmidt_2006, brambilla2017model, Selic_2003, steinberg2008emf, France_2007}}
    \label{tab:mde_characteristics}
\end{table}

\paragraph{Unterschiede zwischen Low-Code und MDE}
Low-Code-Entwicklung und Model-Driven Engineering (MDE) sind zwei Ansätze zur Vereinfachung und Beschleunigung der 
Softwareentwicklung. Obwohl sie ähnliche Ziele verfolgen, unterscheiden sie sich in ihrer Fokussierung, Zielgruppe und Methodik.

Die Low-Code-Entwicklung richtet sich vor allem an sogenannte "Citizen Developers", also Anwender mit wenig 
oder keiner Programmiererfahrung, die dennoch in der Lage sein sollen, Anwendungen zu entwickeln. Dieser Ansatz 
betont die visuelle Programmierung und Benutzerfreundlichkeit, indem er grafische Benutzeroberflächen und Drag-and-Drop-Tools 
bereitstellt. Durch diese visuelle Programmierung wird der Entwicklungsprozess intuitiver und zugänglicher, was es auch 
Nicht-Programmierern ermöglicht, aktiv an der Softwareentwicklung teilzunehmen. Die Hauptstärke der Low-Code-Entwicklung 
liegt in der schnellen und kosteneffizienten Erstellung von Anwendungen, die auf spezifische Geschäftsanforderungen 
zugeschnitten sind. Diese Plattformen sind oft cloud-basiert und bieten integrierte Entwicklungsumgebungen, die den 
gesamten Softwarelebenszyklus unterstützen, von der Anforderungserhebung über das Design bis hin zur Implementierung 
und Wartung \cite{Cabot_2020}.

Im Gegensatz dazu konzentriert sich MDE auf die formale Modellierung und Transformationen. Es richtet sich in erster 
Linie an professionelle Entwickler und Ingenieure, die komplexe technische Projekte realisieren. MDE verwendet formale 
Modelle, um die Struktur und das Verhalten von Software präzise zu beschreiben, und nutzt diese Modelle zur automatischen 
Codegenerierung. Diese Methodik ist besonders nützlich in großen, komplexen Projekten, bei denen eine hohe Präzision und 
Konsistenz erforderlich sind. MDE unterstützt die Erstellung detaillierter und präziser Modelle, die als Grundlage für 
die automatische Generierung von Code dienen, was zu einer höheren Qualität und Konsistenz der Software führt. Werkzeuge 
wie das Eclipse Modeling Framework (EMF) und Acceleo sind Beispiele für MDE-Plattformen, die die Modellierung und die 
Transformation von Modellen in lauffähigen Code unterstützen \cite{di2022low}.

In dieser Arbeit werden die Unterschiede zwischen Low-Code-Entwicklung und MDE zwar kurz behandelt, jedoch sind sie im Kontext 
der vorliegenden Forschungsarbeit von geringerer Relevanz. Der Hauptfokus liegt auf den gemeinsamen Prinzipien und Nutzen 
dieser Ansätze. Beide Ansätze zielen darauf ab, die Softwareentwicklung zu vereinfachen und zu beschleunigen, indem sie 
Abstraktionsebenen einführen und Automatisierung nutzen. Diese Gemeinsamkeiten sind entscheidend für die Entwicklung eines 
Low-Code-Frameworks für Quantencomputing, da sie es ermöglichen, die Komplexität der Quantenprogrammierung zu reduzieren 
und die Zugänglichkeit zu erhöhen. Durch die Integration der Prinzipien sowohl der Low-Code-Entwicklung als auch des MDE 
kann eine innovative Lösung geschaffen werden, die die Vorteile beider Ansätze kombiniert und die Hürden für die Nutzung 
von Quantencomputing senkt.

Für die Ziele dieser Arbeit ist es weniger relevant, ob die Methodik eher auf "Citizen Developers" oder auf professionelle 
Entwickler abzielt. Vielmehr ist die Möglichkeit entscheidend, durch Abstraktion und Automatisierung die Entwicklung 
von Quantenanwendungen zu erleichtern und zu beschleunigen. Die Synergie beider Ansätze könnte dazu beitragen, eine 
breitere Nutzerbasis anzusprechen und die Entwicklungszeit sowie die Kosten erheblich zu reduzieren. Daher konzentriert 
sich diese Arbeit darauf, wie die gemeinsamen Stärken von Low-Code und MDE genutzt werden können, um innovative Lösungen 
im Bereich des Quantencomputings zu entwickeln, anstatt sich auf die spezifischen Unterschiede zu fokussieren.

\paragraph{Gemeinsame Prinzipien und synonymer Gebrauch der Begriffe}
Low-Code-Entwicklung und Model-Driven Engineering (MDE) weisen trotz ihrer Unterschiede in Zielgruppe und Methodik zahlreiche 
gemeinsame Prinzipien auf. Diese gemeinsamen Prinzipien bilden die Grundlage für die synonyme Verwendung der Begriffe in dieser Arbeit.

Ein zentrales gemeinsames Prinzip ist die Abstraktion. Beide Ansätze zielen darauf ab, die Komplexität der Softwareentwicklung 
durch die Einführung höherer Abstraktionsebenen zu reduzieren. In der Low-Code-Entwicklung wird dies durch visuelle Programmierung 
und benutzerfreundliche Oberflächen erreicht, die es auch Nicht-Programmierern ermöglichen, Anwendungen zu erstellen. Im MDE 
wird Abstraktion durch die Verwendung formaler Modelle erreicht, die eine präzise und verständliche Beschreibung der 
Softwarestruktur und -funktionalität bieten. Diese Abstraktion erleichtert die Kommunikation zwischen den verschiedenen 
Stakeholdern und minimiert die Wahrscheinlichkeit von Missverständnissen und Fehlern.

Ein weiteres gemeinsames Prinzip ist die Automatisierung. Sowohl Low-Code-Entwicklung als auch MDE setzen auf die Automatisierung 
von Entwicklungsprozessen, um die Effizienz und Produktivität zu steigern. In der Low-Code-Entwicklung erfolgt dies durch 
Drag-and-Drop-Tools und integrierte Entwicklungsumgebungen, die die manuelle Codierung überflüssig machen. Im MDE wird die 
Automatisierung durch die Transformation von Modellen in ausführbaren Code erreicht, was die Entwicklungszeit verkürzt und die 
Konsistenz zwischen Modell und Code sicherstellt.

Beide Ansätze tragen somit zur Reduzierung der Komplexität in der Softwareentwicklung bei. Durch die Einführung von 
Abstraktionsebenen und die Automatisierung von Entwicklungsprozessen wird die Komplexität der Softwareentwicklung erheblich 
reduziert. Dies führt zu einer besseren Verständlichkeit und Wartbarkeit der entwickelten Anwendungen, was wiederum die 
Qualität der Software verbessert.

In dieser Arbeit werden die Begriffe Low-Code-Entwicklung und Model-Driven Engineering synonym verwendet. Diese synonyme Verwendung 
wird durch die gemeinsamen Ziele und Methoden beider Ansätze begründet. Beide Ansätze streben danach, die Softwareentwicklung 
durch Abstraktion und Automatisierung zu vereinfachen und zu beschleunigen. Sie setzen auf ähnliche Prinzipien, um die Effizienz 
und Produktivität der Entwickler zu steigern und die Komplexität der Entwicklungsprozesse zu reduzieren.

Die Vorteile einer integrativen Betrachtung liegen in der Möglichkeit, die Stärken beider Ansätze zu kombinieren und 
somit eine umfassendere und effektivere Methodik zu entwickeln. Durch die synonyme Verwendung der Begriffe wird es möglich, 
die Prinzipien und Techniken beider Ansätze flexibel anzuwenden und die besten Aspekte beider Welten zu nutzen. Dies ist 
besonders relevant im Kontext der Entwicklung von Quantencomputing-Anwendungen, wo sowohl die Zugänglichkeit für Nicht-Programmierer 
als auch die Präzision und Konsistenz formaler Modelle von entscheidender Bedeutung sind.

Die gemeinsame Betrachtung von Low-Code-Entwicklung und MDE ermöglicht es, innovative Lösungen zu entwickeln, die die Vorteile 
beider Ansätze kombinieren. Dies kann dazu beitragen, die Hürden für die Nutzung von Quantencomputing zu senken und die Entwicklung 
von Anwendungen in diesem Bereich zu beschleunigen. Durch die Nutzung der gemeinsamen Prinzipien von Abstraktion und Automatisierung 
sowie der Reduzierung der Komplexität kann die Softwareentwicklung effizienter und produktiver gestaltet werden.

Zusammenfassend ist in der Tabelle \ref{tab:common_characteristics} eine Übersicht über die gemeinsamen Charakteristika die 
in der vorliegenden Arbeit herangezogen werden, um die Vorteile und Potenziale von Low-Code-Entwicklung 
für die Entwicklung von Quantencomputing-Anwendungen zu verdeutlichen. Diese Charakteristika bilden die Grundlage für die 
weitere Untersuchung der Anwendung von Low-Code-Entwicklung im Bereich des Quantencomputings. 

\begin{table}[h!]
    \centering
    \begin{tabular}{|p{5cm}|p{9cm}|}
    \hline
    \textbf{Charakteristik} & \textbf{Beschreibung} \\
    \hline
    Abstraktion & Einführung höherer Abstraktionsebenen zur Reduzierung der Komplexität \\
    \hline
    Automatisierung & Nutzung von Automatisierung zur Steigerung der Effizienz und Produktivität \\
    \hline
    Reduzierung der Komplexität & Vereinfachung der Softwareentwicklung durch visuelle Programmierung und formale Modellierung \\
    \hline
    Steigerung der Effizienz und Produktivität & Verbesserung der Entwicklungsprozesse durch Minimierung manueller Codierung und Nutzung von Modellen \\
    \hline
    \end{tabular}
    \caption{In dieser Arbeit zugrunde liegende gemeinsame Charakteristika von Low-Code-Entwicklung und Model-Driven Engineering (MDE)}
    \label{tab:common_characteristics}
\end{table}

\section{Quantencomputing}
\begin{itemize}
    \item Quantenmechanik und Qubits
        \begin{itemize}
            \item Einführung in die Quantenmechanik
                \begin{itemize}
                    \item Grundprinzipien: Superposition, Verschränkung, Quanteninterferenz
                    \item Unterschiede zur klassischen Physik
                \end{itemize}
            \item Definition und Eigenschaften von Qubits
                \begin{itemize}
                    \item Unterschied zwischen Bits und Qubits
                    \item Zustände: |0>, |1> und Superpositionen
                \end{itemize}
        \end{itemize}
    \item Quantenschaltkreise und Algorithmen
        \begin{itemize}
            \item Aufbau und Funktionsweise von Quantenschaltkreisen
                \begin{itemize}
                    \item Grundlagen der Quantenschaltungen
                    \item Quantenlogikgatter (Hadamard-Gatter, Pauli-Gatter, CNOT-Gatter)
                    \item Zusammensetzung von Quantenoperationen
                \end{itemize}
            \item Wichtige Quantenalgorithmen
                \begin{itemize}
                    \item Shor-Algorithmus (Faktorisierung)
                    \item Grover-Algorithmus (Datenbanksuche)
                    \item Deutsch-Jozsa-Algorithmus (Bestimmung der Konstanz einer Funktion)
                \end{itemize}
            \item Anwendungsbereiche von Quantenalgorithmen
                \begin{itemize}
                    \item Kryptographie (Quantenkryptographie)
                    \item Optimierungsprobleme (Travelling Salesman Problem)
                    \item Simulation von Quantenmechanischen Systemen (Materialwissenschaften)
                \end{itemize}
        \end{itemize}
    \item Herausforderungen und Potenziale
        \begin{itemize}
            \item Technische Herausforderungen
                \begin{itemize}
                    \item Dekohärenz und Fehlerraten
                    \item Notwendigkeit von Fehlerkorrektur
                    \item Skalierbarkeit von Quantencomputern
                \end{itemize}
            \item Potenziale und Zukunftsperspektiven von Quantencomputing
                \begin{itemize}
                    \item Theoretische Vorteile gegenüber klassischen Computern
                    \item Mögliche Durchbrüche in verschiedenen Forschungs- und Anwendungsgebieten
                    \item Prognosen zur Entwicklung der Quantencomputing-Technologie
                \end{itemize}
        \end{itemize}
        \item Vorteile der Ansätze für Quantencomputing
        \begin{itemize}
            \item Erleichterung der Entwicklung von Quantenalgorithmen
            \item Verbesserung der Zugänglichkeit für nicht-technische Nutzer
            \item Förderung von Innovation und Kollaboration
        \end{itemize}
    \item Herausforderungen und Limitationen bei Nutzung von Low-Code-Entwicklung und MDE
        \begin{itemize}
            \item Technische Herausforderungen
                \begin{itemize}
                    \item Integration in bestehende Quantencomputing-Plattformen
                    \item Skalierbarkeit und Performance
                \end{itemize}
            \item Mögliche Limitationen der Ansätze
                \begin{itemize}
                    \item Begrenzte Anpassungsmöglichkeiten
                    \item Abhängigkeit von spezifischen Plattformen
                \end{itemize}
        \end{itemize}
\end{itemize}

Quantencomputing ist ein aufstrebendes Forschungsgebiet, das sich mit der Entwicklung von
Computern befasst, die auf den Prinzipien der Quantenmechanik basieren. Diese Computer
verwenden Quantenbits oder Qubits, um Informationen zu speichern und zu verarbeiten. Im
Gegensatz zu herkömmlichen Computern, die auf Bits basieren, die entweder den Wert 0 oder
1 haben können, können Qubits gleichzeitig den Wert 0 und 1 haben. Dies ermöglicht es
Quantencomputern, bestimmte Probleme wesentlich schneller zu lösen als herkömmliche
Computer. Quantencomputing hat das Potenzial, eine Vielzahl von Anwendungen zu
revolutionieren, darunter die Kryptographie, die Materialwissenschaft und die
Medikamentenentwicklung.

\paragraph{Open Source}
% \begin{itemize}
%     \item Definition und Bedeutung
%         \begin{itemize}
%             \item Beschreibung von Open-Source-Prinzipien
%                 \begin{itemize}
%                     \item Definition von Open Source
%                     \item Grundprinzipien: Transparenz, Offenheit, Freiheit zur Modifikation und Weiterverbreitung
%                 \end{itemize}
%             \item Geschichte und Entwicklung der Open-Source-Bewegung
%                 \begin{itemize}
%                     \item Ursprünge in der Softwareentwicklung (z.B. GNU-Projekt, Free Software Foundation)
%                     \item Meilensteine in der Open-Source-Geschichte (z.B. Veröffentlichung von Linux, Apache)
%                 \end{itemize}
%             \item Unterschiede zwischen Open Source und proprietärer Software
%                 \begin{itemize}
%                     \item Lizenzierung und Verbreitung
%                     \item Entwicklungsmodelle und Kollaborationsstrategien
%                 \end{itemize}
%         \end{itemize}
%     \item Anwendungen im Low-Code-Framework für Quantencomputing
%         \begin{itemize}
%             \item Beispiele für erfolgreiche Open-Source-Projekte im Bereich Low-Code und Quantencomputing
%                 \begin{itemize}
%                     \item Apache Cordova (Low-Code-Plattform)
%                     \item Qiskit (Open-Source-Quantencomputing-Framework)
%                 \end{itemize}
%             \item Vorteile der Verwendung von Open-Source-Software in der Entwicklung von Quantencomputing-Anwendungen
%                 \begin{itemize}
%                     \item Kosteneffizienz: Reduktion der Softwarekosten
%                     \item Flexibilität: Anpassbarkeit und Erweiterbarkeit des Codes
%                     \item Sicherheit und Stabilität: Durch Peer-Review und Community-Feedback
%                     \item Förderung der Zusammenarbeit und Innovation durch offene Beiträge
%                 \end{itemize}
%             \item Möglichkeiten zur Förderung von Kollaboration und Innovation durch Open-Source-Prinzipien
%                 \begin{itemize}
%                     \item Nutzung von Open-Source-Communities für Feedback und Verbesserung
%                     \item Open-Source-Projekte als Plattform für akademische Forschung und industrielle Anwendungen
%                     \item Beispielprojekte und Case Studies zur Veranschaulichung
%                 \end{itemize}
%         \end{itemize}
%     \item Herausforderungen und Limitationen
%         \begin{itemize}
%             \item Technische und rechtliche Herausforderungen bei der Nutzung und Verbreitung von Open-Source-Software
%                 \begin{itemize}
%                     \item Lizenzierungskonflikte und rechtliche Unsicherheiten
%                     \item Kompatibilitätsprobleme mit proprietärer Software
%                 \end{itemize}
%             \item Risiken und mögliche Nachteile der Open-Source-Entwicklung
%                 \begin{itemize}
%                     \item Mangel an langfristiger Unterstützung und Wartung
%                     \item Sicherheitsrisiken durch offene Angriffsflächen
%                 \end{itemize}
%             \item Strategien zur Bewältigung dieser Herausforderungen
%                 \begin{itemize}
%                     \item Entwicklung klarer Lizenzrichtlinien und Compliance-Programme
%                     \item Aufbau von nachhaltigen Open-Source-Gemeinschaften
%                     \item Einsatz von Sicherheitstools und -praktiken, um potenzielle Schwachstellen zu minimieren
%                 \end{itemize}
%         \end{itemize}
% \end{itemize}
