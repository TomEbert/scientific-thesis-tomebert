Dieser Abschnitt erläutert die grundlegenden Konzepte, die für das Verständnis der Arbeit von zentraler Bedeutung sind: 
Low-Code-Entwicklung, Model-Driven Engineering (MDE) und Quantencomputing. Diese Konzepte bilden das Fundament für die 
Entwicklung eines Low-Code-Werkzeugs, das darauf abzielt, die Anwendung von Quantencomputing zugänglicher und 
benutzerfreundlicher zu gestalten. Im Folgenden werden die genannten Konzepte im Detail dargestellt und ihre Relevanz herausgearbeitet.


\section{Quantencomputing}
Quantencomputing ist ein aufstrebendes Feld der Informatik, das auf den Prinzipien der Quantenmechanik basiert. 
Im Gegensatz zu klassischen Computern, die auf Bits basieren, nutzen Quantencomputer Qubits. 
Diese Fähigkeit beruht auf den quantenmechanischen 
Eigenschaften von Superposition, Verschränkung und Quanteninterferenz, die es Quantencomputern ermöglichen, 
komplexe Berechnungen parallel und effizienter durchzuführen als klassische Computer. Dieser Abschnitt beleuchtet 
die Potenziale des Quantencomputing, die spezifischen Einstiegshürden sowie die 
Herausforderungen, denen sich Entwickler im Vergleich zur klassischen Programmierung gegenübersehen. 

Die Quantenmechanik, die das Verhalten subatomarer Teilchen beschreibt, basiert auf einigen grundlegenden Prinzipien, 
die sie von der klassischen Physik unterscheiden. Eines dieser Prinzipien ist die Superposition, die besagt, dass ein 
Quantenobjekt, wie ein Qubit, sich gleichzeitig in mehreren klassischen Zuständen befinden kann. Während ein klassisches Bit 
entweder 0 oder 1 sein kann, kann ein Qubit in einer Überlagerung dieser Zustände sein, mathematisch dargestellt 
als \( \alpha|0\rangle + \beta|1\rangle \), wobei \( \alpha \) und \( \beta \) komplexe Zahlen sind, die die 
Wahrscheinlichkeitsamplituden repräsentieren~\cite{nielsen2010quantum}.

Ein weiteres zentrales Prinzip der Quantenmechanik ist die Verschränkung, die eine starke Korrelation zwischen den 
Zuständen von zwei oder mehr Quantenobjekten beschreibt. Wenn Qubits verschränkt sind, wird der Zustand eines Qubits 
instantan durch den Zustand des anderen beeinflusst, unabhängig von der Entfernung zwischen ihnen. Diese Eigenschaft 
ermöglicht es, Informationen auf eine Weise zu verarbeiten, die in der klassischen Physik nicht möglich ist~\cite{einstein1935can}.

Quanteninterferenz tritt auf, wenn die Wahrscheinlichkeitsamplituden von Quantenzuständen konstruktiv oder 
destruktiv interferieren. Dies bedeutet, dass sich die Wahrscheinlichkeiten bestimmter Ergebnisse verstärken oder 
abschwächen können. Dieses Phänomen ermöglicht es Quantenalgorithmen, durch gezielte Interferenzen effizientere 
Berechnungen durchzuführen als klassische Algorithmen~\cite{feynman2018simulating}.

In der klassischen Physik sind die Zustände von Systemen deterministisch und können mit absoluter Genauigkeit 
vorhergesagt werden. In der Quantenmechanik sind die Zustände von Systemen probabilistisch, und es kann 
nur die Wahrscheinlichkeit eines bestimmten Zustands vorhergesagt werden~\cite{griffiths2018introduction}. 

Quantenmechanische Effekte wie Verschränkung führen zu Nicht-Lokalität, wobei die Zustände von verschränkten 
Teilchen unabhängig von ihrer Entfernung instantan korreliert sind. In der klassischen Physik gibt es keine 
Entsprechung für diese Art der instantanen Wechselwirkung~\cite{aspect1982experimental}. 

Quantenschaltkreise bilden die Grundlage für Quantenalgorithmen und sind ein zentrales Konzept in der 
Quantencomputing-Entwicklung. Ein Quantenschaltkreis besteht aus einer Sequenz von Quantenlogikgattern, 
die auf Qubits angewendet werden, um Quantenoperationen durchzuführen. Diese Schaltkreise nutzen die 
quantenmechanischen Prinzipien der Superposition und Verschränkung, um parallele Berechnungen durchzuführen, 
die in der klassischen Computerei nicht möglich sind.

Zu den grundlegenden Quantenlogikgattern gehören das Hadamard-Gatter, die Pauli-Gatter und das CNOT-Gatter. 
Das Hadamard-Gatter (H) transformiert ein Qubit von einem Basiszustand in eine Superposition beider 
Basiszustände, mathematisch ausgedrückt als \( H|0\rangle = \frac{1}{\sqrt{2}}(|0\rangle + |1\rangle) \) 
und \( H|1\rangle = \frac{1}{\sqrt{2}}(|0\rangle - |1\rangle) \). Diese Operation ist entscheidend, um 
die quantenmechanische Eigenschaft der Superposition zu nutzen~\cite{griffiths2018introduction}.

Die Pauli-Gatter (X, Y und Z) entsprechen den klassischen NOT-, Y- und Z-Operationen und ändern den 
Zustand eines Qubits entlang der x-, y- oder z-Achse auf der Bloch-Kugel. Zum Beispiel invertiert 
das X-Gatter (auch als Quanten-NOT-Gatter bekannt) den Zustand eines Qubits, 
sodass \( X|0\rangle = |1\rangle \) und \( X|1\rangle = |0\rangle \)~\cite{nielsen2010quantum}.

Das CNOT-Gatter (kontrolliertes NOT) ist ein Zweiqubit Gatter, das eine Verschränkung 
zwischen zwei Qubits erzeugt. Es wirkt auf ein Kontrollqubit und ein Zielqubit. Wenn das Kontrollqubit 
im Zustand \(|1\rangle\) ist, invertiert das CNOT-Gatter den Zustand des Zielqubits. Andernfalls bleibt 
der Zustand des Zielqubits unverändert. 
Mathematisch wird dies als \( CNOT|00\rangle = |00\rangle \), \( CNOT|01\rangle = |01\rangle \), \( CNOT|10\rangle = |11\rangle \) 
und \( CNOT|11\rangle = |10\rangle \) dargestellt. Dieses Gatter ist essenziell für die Implementierung von 
Quantenalgorithmen, die Verschränkungen nutzen~\cite{feynman2018simulating}.

Die Zusammensetzung von Quantenoperationen in einem Quantenschaltkreis erfolgt durch die sequentielle Anwendung 
dieser Gatter auf die Qubits. Die Reihenfolge und Kombination der Gatter bestimmen die durchgeführte 
Quantenoperation und letztendlich das Ergebnis der Berechnung. 
Zum Beispiel besteht der Shor-Algorithmus zur Faktorisierung großer Zahlen aus einer komplizierten Sequenz von 
Quantenoperationen, die Superposition und Verschränkung nutzen, um effizienter zu sein als klassische Algorithmen~\cite{shor1999polynomial}.

Die Entwicklung und Implementierung von Quantenschaltkreisen stellt jedoch erhebliche Herausforderungen 
dar. Die präzise Steuerung und Manipulation von Qubits erfordert spezialisierte Kenntnisse 
die über die klassischen Programmierkenntnisse hinausgehen. Außerdem sind 
Quantenoperationen anfällig für Fehler und Dekohärenz, was die Zuverlässigkeit und Genauigkeit der 
Berechnungen beeinträchtigen kann~\cite{preskill2018quantum}.

Zu den bedeutendsten Quantenalgorithmen zählen der Shor-Algorithmus, der Grover-Algorithmus und der 
Deutsch-Jozsa-Algorithmus. Der Shor-Algorithmus, der zur effizienten Faktorisierung großer Zahlen 
verwendet wird, hat wesentliche Implikationen für die Kryptografie, da viele Verschlüsselungssysteme 
auf der Schwierigkeit der Faktorisierung basieren~\cite{shor1999polynomial}. Der Grover-Algorithmus bietet eine 
quadratische Beschleunigung bei der Suche in unsortierten Datenbanken und kann in Anwendungen mit 
großen Datenmengen von Vorteil sein~\cite{grover1996fast}. Der Deutsch-Jozsa-Algorithmus ermöglicht es, die 
Konstanz einer Funktion exponentiell schneller zu bestimmen als klassische Methoden, und dient als 
Demonstration der Leistungsfähigkeit von Quantenalgorithmen~\cite{deutsch1992rapid}.

In der Kryptografie ermöglicht die Quantenkryptografie die Entwicklung von Kommunikationsprotokollen, die auf den Prinzipien der 
Quantenmechanik basieren und gegenüber Abhörversuchen resistent sind~\cite{bennett2014quantum}. 
Optimierungsprobleme, wie das Travelling Salesman Problem, können durch Quantenalgorithmen effizienter 
gelöst werden, was in der Logistik und anderen Industriezweigen von Nutzen ist~\cite{farhi2000quantum}. 
Zudem ermöglichen Quantenalgorithmen die Simulation von Quantenmechanischen Systemen, was Fortschritte 
in den Materialwissenschaften und der Chemie unterstützen kann~\cite{aspuru2005simulated}. Prognosen zur Entwicklung der Quantencomputing-Technologie 
deuten darauf hin, dass in den kommenden Jahrzehnten signifikante Fortschritte zu erwarten sind~\cite{ladd2010quantum}.

\section{Low-Code-Entwicklung und Model-Driven Engineering}
Low-Code-Entwicklung ist ein Ansatz, der darauf abzielt, die Entwicklung von Anwendungen 
durch die Automatisierung von Prozessen und die Reduzierung der Notwendigkeit von 
handgeschriebenen Code zu vereinfachen. Dieser Ansatz ermöglicht es Fachexperten, Anwendungen 
zu erstellen, ohne umfassende Programmierkenntnisse zu besitzen. Low-Code-Plattformen 
bieten eine Vielzahl von Funktionen, um Anwendungen zu erstellen, zu 
testen und zu implementieren. Dazu gehören unter anderem visuelle Entwicklungsumgebungen, 
die es ermöglichen, Anwendungen durch das grafische Anordnen und Modellieren von Komponenten zu erstellen, 
sowie Datenbanken und anderen Systeme zu integrieren~\cite{Bock_2021}. 
Low-Code-Plattformen bieten auch Funktionen zur Automatisierung von Prozessen, um Anwendungen zu 
erstellen, die auf Ereignisse reagieren und sich an veränderte Bedingungen anpassen können, 
wodurch die Erstellung umfangreicher, datengetriebener Anwendungen vereinfacht wird~\cite{Bock_2021}. 
Shridhar~\cite{Shridhar_2021} und Alamin und Iqbal~\cite{Alamin_2021} zeigen auf, dass diese 
Funktionen nicht nur die Effizienz steigern, sondern auch die Skalierbarkeit und Flexibilität 
der entwickelten Anwendungen erhöhen.

Ein zentrales Merkmal von Low-Code-Plattformen sind visuelle Entwicklungsumgebungen. Diese Umgebungen ermöglichen 
es, Anwendungen durch grafisches Anordnen und Modellieren zu erstellen~\cite{Bock_2021}. Dadurch wird 
der Entwicklungsprozess nicht nur beschleunigt, sondern auch intuitiver gestaltet, sodass auch Nutzer ohne 
tiefgehende Programmierkenntnisse in der Lage sind, komplexe Anwendungen zu entwickeln. Hintergrund ist, dass 
visuelle Darstellungen für Laien einfacher verständlich sind, als Programmiercode. Shridhar~\cite{Shridhar_2021} betont, 
dass diese visuellen Entwicklungswerkzeuge die Barriere für die Softwareentwicklung erheblich senken und die 
Produktivität erhöhen, indem sie die Notwendigkeit, komplexe Programmiersprachen zu benutzen, minimieren.
Low-Code-Plattformen bieten Funktionen, die es ermöglichen, Anwendungen weitgehend ohne traditionellen, 
textbasierten Code zu entwickeln. Stattdessen werden Prozesse und Logiken durch visuelle Modelle und 
Konfigurationen abgebildet~\cite{Bock_2021}. Dies reduziert nicht nur die Fehleranfälligkeit, sondern 
erleichtert auch die Wartung und Anpassung von Anwendungen. Alamin und Iqbal~\cite{Alamin_2021} 
führen aus, dass diese Ansätze Entwickler dabei unterstützen, sich auf die Geschäftslogik und die spezifischen Anforderungen 
der Anwendung zu konzentrieren, anstatt sich mit den Feinheiten der Programmierung auseinanderzusetzen.

Ein weiterer Vorteil der Low-Code-Entwicklung ist die Förderung der Zusammenarbeit zwischen verschiedenen 
Interessensvertretern wie Geschäftsanalysten, Entwickler und Endnutzern. Diese engere Kollaboration führt zu 
einer besseren Übereinstimmung der entwickelten Software mit den geschäftlichen Anforderungen~\cite{Shridhar_2021}. 
Alamin und Iqbal~\cite{Alamin_2021} untersuchen in ihrer empirischen 
Studie die Herausforderungen, denen sich Entwickler in Low-Code-Umgebungen gegenübersehen, und zeigen auf, dass trotz 
der Vorteile auch Aspekte wie Plattformabhängigkeit und eingeschränkte Anpassungsfähigkeit bedacht werden müssen.

Trotz der zahlreichen Vorteile, die Low-Code-Plattformen bieten, sind sie nicht frei von Herausforderungen und Risiken. 
Eine wesentliche Sorge ist die Abhängigkeit von der gewählten Plattform. Diese sogenannte Plattformabhängigkeit kann 
sowohl bei Organisationen als auch bei individuellen Entwicklern dazu führen, dass es schwierig wird, die entwickelten 
Anwendungen auf andere Systeme zu übertragen, falls dies erforderlich wird. Dies könnte langfristige Abhängigkeiten 
von einem bestimmten Anbieter zur Folge haben und die Flexibilität bei zukünftigen Entwicklungen einschränken~\cite{Alamin_2021}. 

Darüber hinaus besteht die Gefahr, dass die Anpassungsmöglichkeiten der von Low-Code-Plattformen generierten 
Anwendungen begrenzt sind. Insbesondere bei spezifischen, nicht standardisierten Anforderungen stoßen solche 
Plattformen oft an ihre Grenzen, was zu suboptimalen Lösungen führen kann~\cite{Alamin_2021}. Auch Sicherheitsaspekte 
können ein Risiko darstellen, da die Verwendung von vorgefertigten Bausteinen potenziell Schwachstellen mit sich 
bringen kann, die nicht immer leicht zu identifizieren sind. Es ist daher unerlässlich, diese potenziellen Risiken 
zu berücksichtigen und sorgfältig abzuwägen, ob die Vorteile die Nachteile überwiegen, insbesondere in spezialisierten 
Anwendungsbereichen wie dem Quantencomputing, wo hohe Flexibilität und maßgeschneiderte Lösungen von besonderer 
Bedeutung sein können. Die wesentlichen Charakteristika der Low-Code-Entwicklung sowie deren Bewertung als Chance oder Risiko, sind in 
Tabelle~\ref{tab:lowcode_characteristics} zusammengefasst.

\begin{table}[h!]
    \centering
    \begin{tabular}{|p{4cm}|p{6cm}|p{4cm}|}
    \hline
    \textbf{Charakteristik} & \textbf{Beschreibung} & \textbf{Chance/Risiko} \\
    \hline
    Visuelle Entwicklungsumgebungen & Anwendungen werden durch grafisches Anordnen und Modellieren von Komponenten erstellt. & Chance: Erhöht die Zugänglichkeit und Benutzerfreundlichkeit \\
    \hline
    Minimierung von textbasiertem Code & Prozesse und Logiken werden durch visuelle Modelle und Konfigurationen abgebildet, ohne traditionellen Code. & Chance: Reduziert Fehleranfälligkeit, Risiko: Eingeschränkte Flexibilität \\
    \hline
    Automatisierung von Prozessen & Anwendungen reagieren auf Ereignisse und passen sich an veränderte Bedingungen an. & Chance: Steigert Effizienz und Anpassungsfähigkeit \\
    \hline
    Integration von Datenbanken und Systemen & Einfachere Erstellung datengetriebener Anwendungen durch Integration mit Datenbanken und anderen Systemen. & Chance: Erhöht die Funktionalität und Datenzugänglichkeit \\
    \hline
    Engere Kollaboration verschiedener Interessensvertreter & Erleichtert die Zusammenarbeit zwischen Geschäftsanalysten, Entwicklern und Endnutzern. & Chance: Verbessert die Übereinstimmung der Software mit den Geschäftsanforderungen \\
    \hline
    Schnellere Entwicklungszeiten & Verkürzt die Entwicklungszeiten durch reduzierte Programmieranforderungen. & Chance: Erhöht die Produktivität \\
    \hline
    Zugänglichkeit für Nicht-Programmierer & Ermöglicht auch Fachexperten ohne tiefgehende Programmierkenntnisse die Entwicklung von Anwendungen. & Chance: Erweitert den Entwicklerkreis \\
    \hline
    Flexibilität und Skalierbarkeit & Erhöht die Flexibilität und Skalierbarkeit der entwickelten Anwendungen. & Chance: Erleichtert Anpassungen und Wachstum \\
    \hline
    Plattformabhängigkeit & Abhängigkeit von spezifischen Low-Code-Plattformen und deren Limitierungen. & Risiko: Erhöht die Abhängigkeit und mögliche Einschränkungen \\
    \hline
    Wartung und Anpassung & Erleichtert die Wartung und Anpassung von Anwendungen durch visuelle Modelle. & Chance: Verbessert die Wartbarkeit und Anpassungsfähigkeit \\
    \hline
    \end{tabular}
    \caption{Charakteristika der Low-Code-Entwicklung und deren Bewertung als Chance oder Risiko~\cite{Bock_2021, Shridhar_2021, Alamin_2021}}
    \label{tab:lowcode_characteristics}
\end{table}

Model-Driven Engineering (MDE) ist ein Ansatz zur Softwareentwicklung, bei dem Modelle als zentrale und treibende 
Elemente des Entwicklungsprozesses fungieren. In MDE dienen diese Modelle nicht nur zur Unterstützung, sondern 
bilden die Grundlage für die Generierung des Quellcodes. Dieser Ansatz ermöglicht es, durch formale Modelle 
eine abstrakte und detaillierte Darstellung der Anwendung und ihrer Funktionalität zu schaffen. Diese Modelle 
werden anschließend zur automatischen Codegenerierung verwendet, was zu einer beschleunigten Entwicklung, einer 
erhöhten Wiederverwendbarkeit und einer verbesserten Qualität der Anwendungen führt~\cite{Selic_2003}.

Ein wesentliches Merkmal von MDE ist die Nutzung formaler Modelle zur Softwareentwicklung. Diese Modelle bieten 
eine präzise und verständliche Beschreibung der Struktur und des Verhaltens der zu entwickelnden 
Software~\cite{Schmidt_2006}. Durch die Abstraktion von technischen Details ermöglichen formale Modelle 
eine klare Kommunikation zwischen den verschiedenen Interessensvertretern im Entwicklungsprozess und tragen dazu bei, 
Missverständnisse zu vermeiden. Formale Modelle können in verschiedenen Formen vorliegen, darunter UML-Diagramme, 
ER-Diagramme und BPMN-Modelle, die jeweils spezifische Aspekte der Software und ihrer Prozesse abbilden~\cite{Selic_2003}.

Ein weiteres zentrales Element von MDE ist die Transformation von Modellen in Code. Dieser Prozess, der als 
Model-to-Code-Transformation bezeichnet wird, ermöglicht es, aus den erstellten Modellen automatisch lauffähigen 
Code zu generieren~\cite{brambilla2017model}. Dies führt nicht nur zu einer Beschleunigung der Entwicklung, sondern 
reduziert auch die Fehleranfälligkeit, da die Notwendigkeit von textbasiertem Code minimiert wird. Durch die Automatisierung der 
Codegenerierung wird zudem die Konsistenz zwischen den Modellen und dem finalen Code sichergestellt. Verschiedene 
Werkzeuge und Plattformen wie das Eclipse Modeling Framework (EMF)~\cite{eclipseEclipse} und 
Acceleo~\cite{eclipseEclipseAcceleo} unterstützen diesen Transformationsprozess 
und ermöglichen eine nahtlose Integration in den Entwicklungsworkflow~\cite{steinberg2008emf}.

MDE fördert außerdem die Wiederverwendbarkeit von Modellen und Komponenten. Durch die Definition wiederverwendbarer 
Modellbausteine können Entwickler effizienter arbeiten und die Qualität der Anwendungen erhöhen. Dies ermöglicht es, 
bewährte Lösungen und Muster zu nutzen und so den Entwicklungsaufwand zu reduzieren~\cite{France_2007}. Zusätzlich 
trägt MDE zur Verbesserung der Wartbarkeit und Erweiterbarkeit von Software bei, da Änderungen am Modell automatisch 
im generierten Code reflektiert werden können.

Trotz der Vorteile gibt es auch Risiken und Herausforderungen, die bei der Anwendung von MDE beachtet werden 
müssen. Zu den Risiken gehört die potenzielle Komplexität der Modelle, die schwer zu verstehen und zu pflegen sein 
können. Darüber hinaus besteht eine Abhängigkeit von speziellen Modellierungs- und Code-Generierungswerkzeugen, was 
die Flexibilität einschränken kann. Die Qualität des automatisch generierten Codes kann variieren und zu 
Performance-Problemen und Fehlern führen. Zudem erfordert MDE spezifische Schulungen und Wissen, was zusätzliche
Kosten und Aufwand bedeutet. Herausforderungen bestehen auch in der Skalierbarkeit und Integration von MDE-Lösungen 
in bestehende Systeme sowie in der Modell-zu-Code-Abstraktionslücke, die dazu führen kann, dass nicht alle Details 
und Optimierungen im generierten Code abgebildet werden~\cite{france2007model}.

Zusammenfassend lässt sich sagen, dass MDE eine umfassende Methodik zur Softwareentwicklung bietet, die durch 
die Nutzung formaler Modelle und die Automatisierung der Codegenerierung sowohl die Effizienz als auch die 
Qualität der entwickelten Anwendungen steigern kann. Die wesentlichen Aspekte und Vorteile von MDE sind 
in Tabelle~\ref{tab:mde_characteristics} zusammengefasst. Diese Übersicht bietet eine klare Struktur zur 
Einordnung der MDE-Methoden, indem sie die verschiedenen Merkmale, deren Beschreibungen und die daraus 
resultierenden Vorteile und Risiken aufzeigt.

\begin{table}[h!]
    \centering
    \begin{tabular}{|p{4cm}|p{6cm}|p{4cm}|}
    \hline
    \textbf{Charakteristik} & \textbf{Beschreibung} & \textbf{Chance/Risiko} \\
    \hline
    Nutzung formaler Modelle & Verwendung von präzisen und verständlichen Modellen zur Beschreibung der Struktur und des Verhaltens der Software. & Chance: Erhöht die Klarheit und Kommunikation zwischen Stakeholdern \\
    \hline
    Transformation von Modellen in Code & Automatische Generierung von lauffähigem Code aus den erstellten Modellen. & Chance: Beschleunigt die Entwicklung und reduziert Fehleranfälligkeit \\
    \hline
    Wiederverwendbarkeit von Modellen & Definition wiederverwendbarer Modellbausteine, um Effizienz und Qualität zu steigern. & Chance: Reduziert Entwicklungsaufwand und erhöht Qualität \\
    \hline
    Verbesserung der Wartbarkeit & Änderungen am Modell werden automatisch im generierten Code reflektiert. & Chance: Erhöht Wartbarkeit und Erweiterbarkeit \\
    \hline
    Komplexität der Modelle & Erstellung sehr komplexer Modelle, die schwer zu verstehen und zu pflegen sind. & Risiko: Erhöht die Komplexität und Wartungsaufwand \\
    \hline
    Werkzeugabhängigkeit & Abhängigkeit von speziellen Modellierungs- und Code-Generierungswerkzeugen. & Risiko: Einschränkung der Flexibilität durch Abhängigkeit von Anbietern \\
    \hline
    Qualität der generierten Codebasis & Variierende Qualität des automatisch generierten Codes. & Risiko: Kann zu Performance-Problemen und Fehlern führen \\
    \hline
    Kosten und Aufwand für Schulungen & Notwendigkeit spezifischer Schulungen und Wissen für die Nutzung von MDE. & Risiko: Erhöht die Kosten und den Aufwand für Schulungen \\
    \hline
    Skalierbarkeit und Integration & Herausforderungen bei der Integration und Skalierung von MDE-Lösungen. & Risiko: Komplexität bei groß angelegten Projekten \\
    \hline
    Modell-zu-Code-Abstraktionslücke & Gefahr, dass Modelle nicht alle Details und Nuancen des Codes abbilden. & Risiko: Fehlen von Details und Optimierungen im generierten Code \\
    \hline
    \end{tabular}
    \caption{Charakteristika des Model-Driven Engineering (MDE), deren Chancen und Risiken~\cite{Schmidt_2006, brambilla2017model, Selic_2003, steinberg2008emf, France_2007}}
    \label{tab:mde_characteristics}
\end{table}

\subsubsection{Unterschiede zwischen Low-Code und MDE}
Low-Code-Entwicklung und Model-Driven Engineering (MDE) sind zwei Ansätze zur Vereinfachung und Beschleunigung der 
Softwareentwicklung. Obwohl sie ähnliche Ziele verfolgen, unterscheiden sie sich in ihrer Fokussierung, Zielgruppe und Methodik.
Die Low-Code-Entwicklung richtet sich primär an sogenannte „Citizen Developers“, also Anwender mit wenig 
oder keiner Programmiererfahrung, die dennoch in der Lage sein wollen oder können, Anwendungen zu entwickeln. Dieser Ansatz zeichnet 
sich durch eine benutzerfreundliche Gestaltung aus, die eine intuitive und visuelle Programmierung ermöglicht. Durch 
grafische Benutzeroberflächen, bei denen Elemente per Maus verschoben und angeordnet werden können, wird der 
Entwicklungsprozess stark vereinfacht, was es Anwendern ohne tiefgreifende Programmierkenntnisse erleichtert, aktiv 
zur Softwareentwicklung beizutragen. Die Hauptstärke der Low-Code-Entwicklung 
liegt in der schnellen und kosteneffizienten Erstellung von Anwendungen, die auf spezifische Geschäftsanforderungen 
zugeschnitten sind. Diese Plattformen sind oft cloud-basiert und bieten integrierte Entwicklungsumgebungen, die den 
gesamten Softwarelebenszyklus unterstützen, von der Anforderungserhebung über das Design bis hin zur Implementierung 
und Wartung~\cite{Cabot_2020}.

Im Gegensatz dazu konzentriert sich MDE auf die formale Modellierung und Transformationen und 
richtet sich vor allem an professionelle Entwickler und Ingenieure. MDE nutzt formale Modelle, um die Struktur und das 
Verhalten von Software präzise zu definieren. Diese Modelle dienen dann als Grundlage für die automatische Codegenerierung, 
was besonders in großen und komplexen Projekten von Vorteil ist, bei denen eine hohe Präzision und Konsistenz entscheidend 
sind. Die Methodik erlaubt es, detaillierte Modelle zu erstellen, die direkt in funktionsfähigen Code umgesetzt werden 
können. Werkzeuge wie das Eclipse Modeling Framework (EMF) und Acceleo unterstützen diesen Prozess, indem sie die Modellierung 
und die Transformation von Modellen in ausführbaren Code ermöglichen~\cite{di2022low}.

In dieser Arbeit werden die Unterschiede zwischen Low-Code-Entwicklung und MDE kurz angesprochen, 
jedoch steht ihre Unterscheidung nicht im Vordergrund der Untersuchung. Der Fokus liegt vielmehr auf den Gemeinsamkeiten beider 
Ansätze und deren Nutzen im spezifischen Kontext der Arbeit. Während die Low-Code-Entwicklung vor allem darauf abzielt, 
auch Nutzer ohne tiefgehende Programmierkenntnisse durch vereinfachte visuelle Programmierung in die Lage zu versetzen, Anwendungen 
zu erstellen, verfolgt MDE das Ziel, die Softwareentwicklung durch formale Modellierung und automatische Codegenerierung zu optimieren 
und zu beschleunigen. Beide Ansätze teilen das übergeordnete Ziel, die Effizienz der Softwareentwicklung zu steigern und 
die Komplexität zu reduzieren.

Diese Gemeinsamkeiten sind von zentraler Bedeutung für die Entwicklung einer Low-Code-Plattform im Bereich Quantencomputing. 
Durch die Reduktion der Programmierkomplexität und die Verbesserung der Zugänglichkeit könnten die Prinzipien beider Ansätze 
dazu beitragen, die Nutzung von Quantencomputing auf eine breitere Anwenderbasis auszudehnen. Dadurch wird eine Lösung angestrebt, 
die die Stärken der Low-Code-Entwicklung in Bezug auf Benutzerfreundlichkeit mit den Vorteilen von MDE in puncto Präzision und 
Automatisierung kombiniert. 

Für die Ziele dieser Arbeit ist es weniger entscheidend, ob die Methodik primär auf „Citizen Developers“ oder professionelle Entwickler abzielt. 
Wesentlich ist vielmehr, dass durch Abstraktion und Automatisierung die Entwicklung von Quantenanwendungen sowohl erleichtert als auch 
beschleunigt wird. Die Synergie beider Ansätze könnte dazu beitragen, eine breitere Nutzerbasis anzusprechen und gleichzeitig die Entwicklungszeit 
signifikant zu verkürzen. Darüber hinaus bieten Low-Code-Entwicklung und MDE Potenzial zur Kosteneinsparung, nicht nur in Bezug auf die 
Entwicklungszeit, sondern auch hinsichtlich der Ressourcen, die für die Implementierung und Wartung der Anwendungen erforderlich sind. 
In dieser Arbeit wird daher der Fokus darauf gelegt, wie die gemeinsamen Stärken von Low-Code und MDE strategisch eingesetzt werden können, 
um effiziente und anpassungsfähige Lösungen im Bereich des Quantencomputing zu entwickeln. Die spezifischen Unterschiede zwischen den 
Ansätzen treten dabei in den Hintergrund zugunsten der Untersuchung ihrer komplementären Eigenschaften.


\subsubsection{Gemeinsame Prinzipien und synonymer Gebrauch der Begriffe}
Low-Code-Entwicklung und Model-Driven Engineering MDE weisen trotz ihrer Unterschiede in Zielgruppe und Methodik zahlreiche 
gemeinsame Prinzipien auf. Diese gemeinsamen Prinzipien bilden die Grundlage für die synonyme Verwendung der Begriffe in dieser Arbeit.

Ein zentrales gemeinsames Prinzip ist die Abstraktion. Beide Ansätze zielen darauf ab, die Komplexität der Softwareentwicklung 
durch die Einführung höherer Abstraktionsebenen zu reduzieren. In der Low-Code-Entwicklung wird dies durch visuelle Programmierung 
und benutzerfreundliche Oberflächen erreicht, die es auch Nicht-Programmierern ermöglichen, Anwendungen zu erstellen. Im MDE 
wird Abstraktion durch die Verwendung formaler Modelle erreicht, die eine präzise und verständliche Beschreibung der 
Softwarestruktur und -funktionalität bieten. Diese Abstraktion erleichtert die Kommunikation zwischen verschiedenen 
Entwicklern eines Projektes und minimiert die Wahrscheinlichkeit von Missverständnissen und Fehlern.

Ein weiteres gemeinsames Prinzip beider Ansätze ist die Automatisierung. Sowohl die Low-Code-Entwicklung als auch  
MDE setzen auf automatisierte Entwicklungsprozesse, um die Effizienz und Produktivität zu steigern. In der Low-Code-Entwicklung 
wird die Automatisierung durch visuelle Entwicklungsumgebungen und grafische Werkzeuge erreicht, die es ermöglichen, ohne umfangreiche 
Programmierkenntnisse Anwendungen zu erstellen. Diese Werkzeuge erlauben es den Nutzern, Softwareelemente mittels einfacher Handhabung 
und grafischer Anordnung zusammenzustellen, wobei diese grafischen Modelle dann automatisch in ausführbaren Code umgewandelt werden. 
Im Gegensatz dazu konzentriert sich MDE auf die Verwendung spezifischer Modelle, die ebenfalls in ausführbaren Code transformiert werden 
können. Dabei ist zu beachten, dass nicht alle MDE-Modelle für die Codegenerierung geeignet sind. Einige Modelle dienen ausschließlich 
der Analyse oder Spezifikation und sind nicht direkt in Code umwandelbar. Die Gemeinsamkeit beider Ansätze liegt hier in der Fähigkeit, 
modellbasierte Spezifikationen in Code zu überführen, wodurch die Entwicklungszeit verkürzt und die Konsistenz zwischen Modell und 
Code gewährleistet wird.

Beide Ansätze tragen durch die Einführung von Abstraktionsebenen und die Automatisierung von Entwicklungsprozessen zur Reduzierung 
der Komplexität in der Softwareentwicklung bei. Dies führt zu einer besseren Verständlichkeit und Wartbarkeit der entwickelten 
Anwendungen, was wiederum die Qualität der Software verbessert.

In dieser Arbeit werden die Begriffe Low-Code-Entwicklung und Model-Driven Engineering ab hier synonym verwendet. Diese synonyme Verwendung 
wird durch die gemeinsamen Ziele und Methoden beider Ansätze begründet. Die gemeinsame Betrachtung von Low-Code-Entwicklung und 
MDE ermöglicht es, Lösungen zu entwickeln, die die Vorteile beider Ansätze kombinieren. Dies kann dazu beitragen, die 
Hürden für die Nutzung von Quantencomputing zu senken und die Entwicklung 
von Anwendungen in diesem Bereich zu beschleunigen. Durch die Nutzung der gemeinsamen Prinzipien von Abstraktion und Automatisierung 
sowie der Reduzierung der Komplexität kann die Softwareentwicklung effizienter und produktiver gestaltet werden.

Zusammenfassend ist in der Tabelle~\ref{tab:common_characteristics} eine Übersicht über die gemeinsamen Charakteristika, die 
in der vorliegenden Arbeit herangezogen werden, um die Vorteile und Potenziale von Low-Code-Entwicklung 
für die Entwicklung von Quantencomputing-Anwendungen zu verdeutlichen. Diese Charakteristika bilden die Grundlage für die 
weitere Untersuchung der Anwendung von Low-Code-Entwicklung im Bereich des Quantencomputing. 

\begin{table}[h!]
    \centering
    \begin{tabular}{|p{5cm}|p{9cm}|}
    \hline
    \textbf{Charakteristik} & \textbf{Beschreibung} \\
    \hline
    Abstraktion & Einführung höherer Abstraktionsebenen zur Reduzierung der Komplexität \\
    \hline
    Automatisierung & Nutzung von Automatisierung zur Steigerung der Effizienz und Produktivität \\
    \hline
    Reduzierung der Komplexität & Vereinfachung der Softwareentwicklung durch visuelle Programmierung und formale Modellierung \\
    \hline
    Steigerung der Effizienz und Produktivität & Verbesserung der Entwicklungsprozesse durch Minimierung manueller Codierung und Nutzung von Modellen \\
    \hline
    \end{tabular}
    \caption{In dieser Arbeit zugrunde liegende gemeinsame Charakteristika von Low-Code-Entwicklung und MDE}
    \label{tab:common_characteristics}
\end{table}

\subsubsection{Open Source}
Ein weiterer wichtiger Aspekt in der Betrachtung von Low-Code und MDE ist der 
Einsatz von Open Source Lösungen. Open-Source-Software bietet zahlreiche Vorteile, darunter die Möglichkeit 
zur gemeinschaftlichen Entwicklung und kontinuierlichen Verbesserung durch eine breite Entwicklergemeinde. 
Dieser kollaborative Ansatz fördert eine schnelle Identifizierung und Behebung 
von Fehlern. Zudem sind Open Source Lösungen in der Regel kostengünstiger, da sie keine Lizenzgebühren 
erfordern, was insbesondere für Forschungsprojekte und kleinere Unternehmen von Vorteil 
ist~\cite{raymond2010cathedral}. Open Source Plattformen wie GitHub und GitLab bieten umfangreiche Werkzeuge und 
Ressourcen, die die Entwicklung und Verbreitung von Softwareprojekten erleichtern~\cite{fitzgerald2006transformation}.

Darüber hinaus ermöglichen Open Source Lösungen eine hohe Transparenz und Sicherheit. Der offene Zugang 
zum Quellcode erlaubt es den Nutzern, den Code zu überprüfen, anzupassen und an spezifische Bedürfnisse 
anzupassen. Dies erhöht nicht nur die Sicherheit durch unabhängige Überprüfungen, sondern bietet auch 
die Flexibilität, individuelle Anpassungen vorzunehmen, die in proprietären Systemen oft nicht 
möglich sind~\cite{von2006promise}.

Allerdings gibt es auch Limitationen. Open-Source-Projekte können unter mangelnder Unterstützung und 
Dokumentation leiden, was die Lernkurve für neue Nutzer steiler macht. Zudem ist die Integration von 
Open Source Komponenten in bestehende Systeme oft komplex und erfordert spezifisches technisches 
Know-how~\cite{chesbrough2006open}. Während die breite Verfügbarkeit von Open-Source-Projekten eine große 
Auswahl bietet, kann dies auch zu einer Fragmentierung führen, bei der es schwierig ist, eine konsistente 
und stabile Lösung zu finden und zu implementieren.

Trotz dieser Herausforderungen bieten Open Source Lösungen eine flexible und zugängliche Plattform für 
die Entwicklung und Implementierung von Low-Code und MDE Ansätzen im Quantencomputing, 
die durch die gemeinsame Nutzung von Wissen und Ressourcen gestärkt 
wird. Dies ist besonders relevant in einem sich schnell entwickelnden und komplexen Feld wie dem 
Quantencomputing, wo der Zugang zu neuesten Entwicklungen und die Fähigkeit zur schnellen Anpassung 
entscheidend sind. Die Nutzung von Open Source Lösungen kann daher nicht nur die Entwicklung 
beschleunigen, sondern auch die Zusammenarbeit und den Wissensaustausch zwischen verschiedenen 
Disziplinen und Institutionen fördern~\cite{west2006challenges}.

\section{Low-Code-Entwicklung und Model-Driven Engineering im Kontext des Quantencomputing}
Im Kontext des Quantencomputing bedeutet dies für die Entwicklung von Low-Code-Werkzeugen, dass diese eine 
benutzerfreundliche Oberfläche bieten müssen, die es Entwicklern ermöglicht, komplexe Quantenschaltungen 
zu erstellen, ohne tiefgehende Kenntnisse der Quantenmechanik zu benötigen. Diese Tools sollten Mechanismen 
zur Fehlerkorrektur und zur Optimierung der Quantenschaltungen beinhalten, um die Zuverlässigkeit und 
Effizienz der Quantenoperationen zu gewährleisten. 

Die Einführung von Low-Code im Quantencomputing könnte potenziell mehrere bedeutende Vorteile bieten. 
Ein wesentlicher Vorteil ist die Erleichterung der Entwicklung von Quantenalgorithmen. Durch die 
Bereitstellung von benutzerfreundlichen, visuellen Entwicklungsumgebungen, die die komplexen 
quantenmechanischen Prinzipien abstrahieren, können Low-Code-Werkzeuge es Entwicklern ermöglichen, 
Quantenalgorithmen effizienter zu erstellen und zu testen. Dies könnte insbesondere für Entwickler nützlich 
ein, die über begrenzte Kenntnisse in der Quantenmechanik verfügen, da sie sich auf die Logik und 
Funktionalität ihrer Algorithmen konzentrieren können, ohne tief in die physikalischen Details 
eintauchen zu müssen~\cite{Cabot_2020}.

Ein weiterer Vorteil ist die Verbesserung der Zugänglichkeit für Citizen Developer. Low-Code-Werkzeuge 
könnten die Barrieren senken. Durch intuitive Schnittstellen und vorgefertigte Module könnten auch Nutzer 
aus anderen Disziplinen, wie Wirtschaft, Biologie oder Chemie, Quantenalgorithmen entwickeln und 
anwenden, um spezifische Probleme in ihren Bereichen zu lösen~\cite{di2022low}. Dies könnte zu 
einer breiteren Akzeptanz und Nutzung von Quantencomputing führen und neue Anwendungen ermöglichen.

Dennoch gibt es Herausforderungen und Limitationen bei der Nutzung von Low-Code-Entwicklung und 
MDE im Quantencomputing. Eine der größten technischen Herausforderungen 
besteht in der Integration dieser Tools in bestehende Quantencomputing-Plattformen. Die Komplexität 
der Quantenhardware und -software erfordert spezialisierte Schnittstellen und Protokolle, um eine 
nahtlose Integration zu gewährleisten~\cite{monroe2013scaling}.

Mögliche Limitationen der Low-Code-Ansätze umfassen begrenzte Anpassungsmöglichkeiten und die 
Abhängigkeit von spezifischen Plattformen. Low-Code-Werkzeuge bieten in der Regel vordefinierte 
Module und Bausteine, die zwar die Entwicklung erleichtern, aber möglicherweise nicht die 
Flexibilität bieten, die für spezialisierte oder fortgeschrittene Quantenanwendungen erforderlich 
ist. Zudem könnten diese Tools von den spezifischen Technologien und Plattformen abhängen, auf denen 
sie entwickelt wurden, was ihre Portabilität und Interoperabilität einschränken könnte~\cite{nielsen2010quantum}.

Die Tabelle~\ref{tab:lowcode_potentials_challenges} fasst die potenziellen Vorteile der 
Low-Code-Entwicklung im Quantencomputing sowie die damit verbundenen Herausforderungen und Hindernisse 
zusammen, um einen umfassenden Überblick über die Chancen und die technischen Hürden zu bieten, die 
bei der Implementierung solcher Ansätze berücksichtigt werden müssen.

\begin{table}[h!]
    \centering
    \begin{tabular}{|p{7cm}|p{7cm}|}
    \hline
    \textbf{Potenziale von Low-Code im Quantencomputing} & \textbf{Herausforderungen und Hindernisse} \\
    \hline
    Erleichterung der Entwicklung von Quantenalgorithmen durch benutzerfreundliche, visuelle Entwicklungsumgebungen & Integration in bestehende Quantencomputing-Plattformen \\
    \hline
    Verbesserung der Zugänglichkeit für nicht-technische Nutzer durch intuitive Schnittstellen und vorgefertigte Module & Skalierbarkeit und Performance bei großen Quantenoperationen \\
    \hline
    Förderung von Kollaboration durch interdisziplinäre Teams und breitere Nutzung & Begrenzte Anpassungsmöglichkeiten bei spezialisierten oder fortgeschrittenen Anwendungen \\
    \hline
    Beschleunigung der Entwicklungsprozesse durch Automatisierung und Abstraktion der komplexen quantenmechanischen Prinzipien & Abhängigkeit von spezifischen Plattformen und Technologien \\
    \hline
    Reduzierung der Einstiegshürden für neue Entwickler im Bereich Quantencomputing & Notwendigkeit von Fehlerkorrektur und Management von Dekohärenz und Fehlerraten \\
    \hline
    \end{tabular}
    \caption{Potenziale und Herausforderungen von Low-Code im Quantencomputing}
    \label{tab:lowcode_potentials_challenges}
\end{table}