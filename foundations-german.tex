\section{Grundlagen}
Die vorliegende Bachelorarbeit widmet sich der Konzeption und Entwicklung eines 
Low-Code-Frameworks für Quantencomputing-Anwendungen. Sie zielt darauf ab, die 
Brücke zwischen der Komplexität des Quantencomputings und der Zugänglichkeit durch 
Low-Code-Ansätze zu schlagen. 

\subsection{Systematischer Literaturreview}
Basierend auf Kitchenham und Charters \cite{kitchenham2007guidelines} wird ein systematischer 
Literaturreview (SLR) durchgeführt, um ein Verständnis der gemeinsamen Charakteristika, Möglichkeiten 
und Hindernissen von Low-Code-Prinzipien zu erlangen. Die SLR bietet einen bewährten methodischen 
Rahmen für systematische Literaturreviews in der Softwaretechnik. Basierend darauf wird die 
anschließende Integration in die Entwicklung von Quantencomputing-Anwendungen aufgebaut.

\paragraph{Methodik}
Der SLR beginnt mit der Formulierung präziser Recherchefragen, die sich auf die 
Identifikation und Analyse bestehender Low-Code-Plattformen, deren Programmiersprachen, 
den spezifischen Fokus und Herausforderungen im Kontext des Quantencomputings 
konzentrieren. Ein detailliertes Suchprotokoll legt relevante Datenbanken und 
Suchbegriffe fest, während Einschluss- und Ausschlusskriterien die Relevanz und Qualität 
der Studien gewährleisten.

\paragraph{Datenextraktion und Synthese}
Aus den selektierten Studien werden systematisch Daten zu den Kernaspekten der 
Low-Code-Plattformen extrahiert. Diese Daten werden anschließend zusammengeführt, um ein 
umfassendes Bild der aktuellen Forschungslandschaft und potenzieller Entwicklungspfade für 
das geplante Framework zu zeichnen.

\paragraph{Festgelegte Kriterien des SLR}
In den Suchmaschinen \textit{Google Scholar}, \textit{IEEE Xplore} und \textit{ACM Digital Library} 
wird gesucht. Die Auswahl der Suchbegriffe wird strategisch vorgenommen, um die Breite und Tiefe der beiden 
Hauptthemenbereiche abzudecken. 

Zu den primären Suchbegriffen gehören 
\textit{Low-Code Development}, \textit{Quantum Computing}, \textit{Model-Driven Engineering (MDE)}, 
\textit{Visual Programming Languages}, und \textit{Quantum Software Development}. Des Weiteren 
werden Suchbegriffe wie \textit{Quantum Circuits Design}, \textit{Open Source Low-Code Platforms} 
\textit{Open Source Low-Code Development Tools} und
\textit{Integration of Quantum Algorithms} ausgewählt, um spezifische Aspekte der 
Anwendungsentwicklung im Quantencomputing sowie mögliche Open-Source Platformen 
zu beleuchten. Die Herausforderungen, die bei 
der Entwicklung von Quantencomputing-Anwendungen auftreten, werden durch Begriffe wie 
\textit{Challenges in Quantum Computing Development} adressiert. Schließlich soll die 
Einbeziehung von \textit{Case Studies on Low-Code Applications} helfen, praktische Anwendungsbeispiele 
und Erfolgsgeschichten zu identifizieren, die für die Konzeption des Low-Code-Frameworks für 
Quantencomputing-Anwendungen relevant sein könnten.

Um die Schnittstellen zwischen Low-Code-Plattformen und Quantencomputing genauer zu untersuchen, 
wurden die Suchbegriffe in verschiedenen Kombinationen verwendet, wobei Boolesche Operatoren wie 
AND und OR zum Einsatz kamen, um die Suche zu verfeinern und zu spezifizieren.

Die Studien müssen frühestens ab dem Jahr 2014 
veröffentlicht worden sein. Es werden nur Studien in englischer oder deutscher 
Sprache berücksichtigt. Weiterhin werden nur Studien berücksichtigt, die Peer-Review-Verfahren 
durchlaufen haben. Insbesondere als relevant werden Papers erachtet, die Low-Code-Plattformen 
in Open-Source Lizensierungen behandeln. Dies ermöglicht eine vereinfachte Verwendbarkeit für 
die später folgende pilothafte Umsetzung eines Low-Code Tools für Quantencomputing-Anwendungen.

Durch die Anwendung dieser 
methodischen und strukturierten Vorgehensweise wird eine solide Basis für die Erforschung 
und Analyse der Konzeption und Entwicklung von Low-Code-Frameworks für 
Quantencomputing-Anwendungen geschaffen.

\subsection{Low-Code-Entwicklung}
Low-Code-Entwicklung ist ein Ansatz, der darauf abzielt, die Entwicklung von Anwendungen
durch die Automatisierung von Prozessen und die Reduzierung der Notwendigkeit von
Handcodierung zu vereinfachen. Dieser Ansatz ermöglicht es Fachexperten, Anwendungen
zu erstellen, ohne umfassende Programmierkenntnisse zu besitzen. Low-Code-Plattformen
bieten eine Vielzahl von Funktionen, die es ermöglichen, Anwendungen zu erstellen, zu
testen und zu implementieren. Dazu gehören unter anderem visuelle Entwicklungsumgebungen,
die es ermöglichen, Anwendungen durch das Ziehen und Ablegen von Komponenten zu erstellen,
sowie die Integration von Datenbanken und anderen Systemen. Low-Code-Plattformen bieten
auch Funktionen zur Automatisierung von Prozessen, die es ermöglichen, Anwendungen zu
erstellen, die auf Ereignisse reagieren und sich an veränderte Bedingungen anpassen können.

\subsection{Model-Driven Engineering}
Model-Driven Engineering (MDE) ist ein Ansatz zur Softwareentwicklung, der darauf abzielt,
die Entwicklung von Anwendungen durch die Verwendung von Modellen zu vereinfachen. MDE
ermöglicht es Entwicklern, Anwendungen zu erstellen, indem sie Modelle verwenden, die
eine abstrakte Darstellung der Anwendung und ihrer Funktionalität darstellen. Diese
Modelle können dann verwendet werden, um den Code für die Anwendung automatisch zu
generieren. MDE bietet eine Reihe von Vorteilen, darunter die Möglichkeit, Anwendungen
schneller zu entwickeln, die Wiederverwendung von Modellen und die Möglichkeit, die
Qualität der Anwendungen zu verbessern.

\subsection{Quantencomputing}
Quantencomputing ist ein aufstrebendes Forschungsgebiet, das sich mit der Entwicklung von
Computern befasst, die auf den Prinzipien der Quantenmechanik basieren. Diese Computer
verwenden Quantenbits oder Qubits, um Informationen zu speichern und zu verarbeiten. Im
Gegensatz zu herkömmlichen Computern, die auf Bits basieren, die entweder den Wert 0 oder
1 haben können, können Qubits gleichzeitig den Wert 0 und 1 haben. Dies ermöglicht es
Quantencomputern, bestimmte Probleme wesentlich schneller zu lösen als herkömmliche
Computer. Quantencomputing hat das Potenzial, eine Vielzahl von Anwendungen zu
revolutionieren, darunter die Kryptographie, die Materialwissenschaft und die
Medikamentenentwicklung.


