\section{Grundlagen}
Die vorliegende Bachelorarbeit widmet sich der Konzeption und Entwicklung eines 
Low-Code-Frameworks für Quantencomputing-Anwendungen. Sie zielt darauf ab, die 
Brücke zwischen der Komplexität des Quantencomputings und der Zugänglichkeit durch 
Low-Code-Ansätze zu schlagen. 

\section{Low-Code-Entwicklung}
\begin{itemize}
    \item Definition und Prinzipien
        \begin{itemize}
            \item Beschreibung, was Low-Code-Entwicklung ist
            \item Grundlegende Prinzipien von Low-Code-Plattformen
            \item Unterschiede zu traditionellen Entwicklungsansätzen
        \end{itemize}
    \item Vorteile und Herausforderungen
        \begin{itemize}
            \item Vorteile
                \begin{itemize}
                    \item Schnellere Entwicklungszeiten
                    \item Geringerer Bedarf an Programmierkenntnissen
                    \item Erhöhte Zugänglichkeit für Nicht-Programmierer
                    \item Kosteneinsparungen
                \end{itemize}
            \item Herausforderungen
                \begin{itemize}
                    \item Begrenzte Anpassungsmöglichkeiten
                    \item Abhängigkeit von Plattformanbietern (Vendor Lock-in)
                    \item Performance-Einschränkungen
                    \item Sicherheitsaspekte
                \end{itemize}
        \end{itemize}
\end{itemize}

Low-Code-Entwicklung ist ein Ansatz, der darauf abzielt, die Entwicklung von Anwendungen
durch die Automatisierung von Prozessen und die Reduzierung der Notwendigkeit von
Handcodierung zu vereinfachen. Dieser Ansatz ermöglicht es Fachexperten, Anwendungen
zu erstellen, ohne umfassende Programmierkenntnisse zu besitzen. Low-Code-Plattformen
bieten eine Vielzahl von Funktionen, die es ermöglichen, Anwendungen zu erstellen, zu
testen und zu implementieren. Dazu gehören unter anderem visuelle Entwicklungsumgebungen,
die es ermöglichen, Anwendungen durch das Ziehen und Ablegen von Komponenten zu erstellen,
sowie die Integration von Datenbanken und anderen Systemen. Low-Code-Plattformen bieten
auch Funktionen zur Automatisierung von Prozessen, die es ermöglichen, Anwendungen zu
erstellen, die auf Ereignisse reagieren und sich an veränderte Bedingungen anpassen können.

\section{Model-Driven Engineering}
\begin{itemize}
    \item Definition und Prinzipien
        \begin{itemize}
            \item Beschreibung von Model-Driven Engineering (MDE)
            \item Grundlegende Prinzipien und Konzepte von MDE
            \item Unterschied zu traditionellen Softwareentwicklungsansätzen
        \end{itemize}
    \item Anwendungen im Low-Code-Framework für Quantencomputing
        \begin{itemize}
            \item Einsatz von MDE zur Modellierung von Quantenalgorithmen
            \item Beispiele für die Integration von MDE in Low-Code-Plattformen
            \item Vorteile von MDE in der Entwicklung von Quantencomputing-Anwendungen
        \end{itemize}
    \item Herausforderungen und Limitationen
        \begin{itemize}
            \item Technische Herausforderungen bei der Implementierung von MDE im Quantencomputing
            \item Begrenzungen durch die Komplexität der Modellierung von Quantenphänomenen
            \item Mögliche Lösungen und Ansätze zur Überwindung dieser Herausforderungen
        \end{itemize}
\end{itemize}

Model-Driven Engineering (MDE) ist ein Ansatz zur Softwareentwicklung, der darauf abzielt,
die Entwicklung von Anwendungen durch die Verwendung von Modellen zu vereinfachen. MDE
ermöglicht es Entwicklern, Anwendungen zu erstellen, indem sie Modelle verwenden, die
eine abstrakte Darstellung der Anwendung und ihrer Funktionalität darstellen. Diese
Modelle können dann verwendet werden, um den Code für die Anwendung automatisch zu
generieren. MDE bietet eine Reihe von Vorteilen, darunter die Möglichkeit, Anwendungen
schneller zu entwickeln, die Wiederverwendung von Modellen und die Möglichkeit, die
Qualität der Anwendungen zu verbessern.

\section{Quantencomputing}
\begin{itemize}
    \item Quantenmechanik und Qubits
        \begin{itemize}
            \item Einführung in die Quantenmechanik
                \begin{itemize}
                    \item Grundprinzipien: Superposition, Verschränkung, Quanteninterferenz
                    \item Unterschiede zur klassischen Physik
                \end{itemize}
            \item Definition und Eigenschaften von Qubits
                \begin{itemize}
                    \item Unterschied zwischen Bits und Qubits
                    \item Darstellung von Qubits (Bloch-Kugel, Bra-Ket-Notation)
                    \item Zustände: |0>, |1> und Superpositionen
                \end{itemize}
        \end{itemize}
    \item Quantenschaltkreise und Algorithmen
        \begin{itemize}
            \item Aufbau und Funktionsweise von Quantenschaltkreisen
                \begin{itemize}
                    \item Grundlagen der Quantenschaltungen
                    \item Quantenlogikgatter (Hadamard-Gatter, Pauli-Gatter, CNOT-Gatter)
                    \item Zusammensetzung von Quantenoperationen
                \end{itemize}
            \item Wichtige Quantenalgorithmen
                \begin{itemize}
                    \item Shor-Algorithmus (Faktorisierung)
                    \item Grover-Algorithmus (Datenbanksuche)
                    \item Deutsch-Jozsa-Algorithmus (Bestimmung der Konstanz einer Funktion)
                \end{itemize}
            \item Anwendungsbereiche von Quantenalgorithmen
                \begin{itemize}
                    \item Kryptographie (Quantenkryptographie)
                    \item Optimierungsprobleme (Travelling Salesman Problem)
                    \item Simulation von Quantenmechanischen Systemen (Materialwissenschaften)
                \end{itemize}
        \end{itemize}
    \item Herausforderungen und Potenziale
        \begin{itemize}
            \item Technische Herausforderungen
                \begin{itemize}
                    \item Dekohärenz und Fehlerraten
                    \item Notwendigkeit von Fehlerkorrektur
                    \item Skalierbarkeit von Quantencomputern
                \end{itemize}
            \item Potenziale und Zukunftsperspektiven von Quantencomputing
                \begin{itemize}
                    \item Theoretische Vorteile gegenüber klassischen Computern
                    \item Mögliche Durchbrüche in verschiedenen Forschungs- und Anwendungsgebieten
                    \item Prognosen zur Entwicklung der Quantencomputing-Technologie
                \end{itemize}
        \end{itemize}
\end{itemize}

Quantencomputing ist ein aufstrebendes Forschungsgebiet, das sich mit der Entwicklung von
Computern befasst, die auf den Prinzipien der Quantenmechanik basieren. Diese Computer
verwenden Quantenbits oder Qubits, um Informationen zu speichern und zu verarbeiten. Im
Gegensatz zu herkömmlichen Computern, die auf Bits basieren, die entweder den Wert 0 oder
1 haben können, können Qubits gleichzeitig den Wert 0 und 1 haben. Dies ermöglicht es
Quantencomputern, bestimmte Probleme wesentlich schneller zu lösen als herkömmliche
Computer. Quantencomputing hat das Potenzial, eine Vielzahl von Anwendungen zu
revolutionieren, darunter die Kryptographie, die Materialwissenschaft und die
Medikamentenentwicklung.

\section{Open Source}
\begin{itemize}
    \item Definition und Bedeutung
        \begin{itemize}
            \item Beschreibung von Open-Source-Prinzipien
                \begin{itemize}
                    \item Definition von Open Source
                    \item Grundprinzipien: Transparenz, Offenheit, Freiheit zur Modifikation und Weiterverbreitung
                \end{itemize}
            \item Geschichte und Entwicklung der Open-Source-Bewegung
                \begin{itemize}
                    \item Ursprünge in der Softwareentwicklung (z.B. GNU-Projekt, Free Software Foundation)
                    \item Meilensteine in der Open-Source-Geschichte (z.B. Veröffentlichung von Linux, Apache)
                \end{itemize}
            \item Unterschiede zwischen Open Source und proprietärer Software
                \begin{itemize}
                    \item Lizenzierung und Verbreitung
                    \item Entwicklungsmodelle und Kollaborationsstrategien
                \end{itemize}
        \end{itemize}
    \item Anwendungen im Low-Code-Framework für Quantencomputing
        \begin{itemize}
            \item Beispiele für erfolgreiche Open-Source-Projekte im Bereich Low-Code und Quantencomputing
                \begin{itemize}
                    \item Apache Cordova (Low-Code-Plattform)
                    \item Qiskit (Open-Source-Quantencomputing-Framework)
                \end{itemize}
            \item Vorteile der Verwendung von Open-Source-Software in der Entwicklung von Quantencomputing-Anwendungen
                \begin{itemize}
                    \item Kosteneffizienz: Reduktion der Softwarekosten
                    \item Flexibilität: Anpassbarkeit und Erweiterbarkeit des Codes
                    \item Sicherheit und Stabilität: Durch Peer-Review und Community-Feedback
                    \item Förderung der Zusammenarbeit und Innovation durch offene Beiträge
                \end{itemize}
            \item Möglichkeiten zur Förderung von Kollaboration und Innovation durch Open-Source-Prinzipien
                \begin{itemize}
                    \item Nutzung von Open-Source-Communities für Feedback und Verbesserung
                    \item Open-Source-Projekte als Plattform für akademische Forschung und industrielle Anwendungen
                    \item Beispielprojekte und Case Studies zur Veranschaulichung
                \end{itemize}
        \end{itemize}
    \item Herausforderungen und Limitationen
        \begin{itemize}
            \item Technische und rechtliche Herausforderungen bei der Nutzung und Verbreitung von Open-Source-Software
                \begin{itemize}
                    \item Lizenzierungskonflikte und rechtliche Unsicherheiten
                    \item Kompatibilitätsprobleme mit proprietärer Software
                \end{itemize}
            \item Risiken und mögliche Nachteile der Open-Source-Entwicklung
                \begin{itemize}
                    \item Mangel an langfristiger Unterstützung und Wartung
                    \item Sicherheitsrisiken durch offene Angriffsflächen
                \end{itemize}
            \item Strategien zur Bewältigung dieser Herausforderungen
                \begin{itemize}
                    \item Entwicklung klarer Lizenzrichtlinien und Compliance-Programme
                    \item Aufbau von nachhaltigen Open-Source-Gemeinschaften
                    \item Einsatz von Sicherheitstools und -praktiken, um potenzielle Schwachstellen zu minimieren
                \end{itemize}
        \end{itemize}
\end{itemize}
