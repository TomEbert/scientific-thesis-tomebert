\paragraph{Supporting the understanding and comparison of low-code development platforms}

Das Paper von Apurvanand Sahay et al. (2020)~\cite{Sahay_2020} untersucht verschiedene Low-Code-Entwicklungsplattformen und vergleicht deren 
Eigenschaften und Fähigkeiten. Zu den untersuchten Plattformen gehören OutSystems, Mendix, Appian, PowerApps und Salesforce. 
Diese Plattformen werden hinsichtlich ihrer Benutzerfreundlichkeit, Anpassungsfähigkeit und Integration in bestehende Systeme analysiert.

Das Paper diskutiert ausführlich die folgenden Low-Code-Entwicklungsplattformen: OutSystems, Mendix, Appian, PowerApps und Salesforce. 
Diese Tools werden hinsichtlich ihrer Funktionen und Einsatzmöglichkeiten im klassischen Softwareentwicklungsprozess untersucht.
Möglichkeiten der Anwendung im Quantencomputing werden in diesem Paper nicht erwähnt. Der Fokus liegt ausschließlich auf klassischen 
Low-Code-Plattformen und deren Vergleich.
Das Paper wurde im Jahr 2020 veröffentlicht und behandelt die zu diesem Zeitpunkt aktuellen Konzepte und Tools. Die behandelten 
Plattformen und deren Funktionen spiegeln den Stand der Technik zur Zeit der Veröffentlichung wider.
Die im Paper erwähnten Low-Code-Plattformen sind kommerzielle Produkte. Es wird nicht auf Open-Source-Alternativen 
eingegangen, sondern auf etablierte, proprietäre Lösungen.

\paragraph{In Search of the Essence of Low-Code: An Exploratory Study of Seven Development Platforms}

Das Paper von Alexander C. Bock (2021)~\cite{Bock_2021_essence} bietet eine tiefgehende Analyse von sieben verschiedenen Low-Code-Entwicklungsplattformen. 
Es untersucht die Kernmerkmale dieser Plattformen und beleuchtet deren Einsatzmöglichkeiten sowie Einschränkungen. Die analysierten 
Plattformen umfassen Betty Blocks, GeneXus, Quick Base, OutSystems, Mendix, PowerApps und Salesforce.

Es werden die Kernmerkmale, Einsatzmöglichkeiten und Einschränkungen dieser Plattformen detailliert beschrieben. 
Möglichkeiten der Anwendung im Quantencomputing werden in diesem Paper ebenfalls nicht behandelt. 
Der Fokus liegt auf der Analyse und dem Vergleich der Plattformen im Kontext klassischer Softwareentwicklungsprozesse. 
Die Studie wurde im Jahr 2021 durchgeführt und behandelt aktuelle Konzepte und Tools zur Zeit der Veröffentlichung. 
Die im Paper untersuchten Low-Code-Plattformen sind sowohl kommerzielle als auch proprietäre Lösungen, 
wobei Open-Source-Alternativen nicht im Fokus stehen.

\paragraph{What about the usability in low-code platforms? A systematic literature review}

Das Paper von Daniel Pinho und Vasco Amaral (2022)~\cite{Pinho_2022} bietet eine systematische Literaturübersicht zur Benutzerfreundlichkeit von 
Low-Code-Plattformen. Es untersucht die wichtigsten Kriterien und Herausforderungen im Zusammenhang mit der Usability 
dieser Plattformen und analysiert eine Vielzahl von bestehenden Studien, um ein umfassendes Bild der aktuellen Forschungslage zu zeichnen.

Das Paper identifiziert und bewertet mehrere Low-Code-Entwicklungsplattformen, allerdings werden spezifische Plattformen 
und Tools nicht namentlich genannt. Der Schwerpunkt liegt auf der Analyse von Benutzerfreundlichkeit und den damit 
verbundenen Herausforderungen, wie zum Beispiel der Lernkurve für neue Nutzer und der Anpassungsfähigkeit der 
Plattformen an unterschiedliche Nutzerbedürfnisse. 
Möglichkeiten der Anwendung im Quantencomputing werden in diesem Paper nicht diskutiert. 
Die Untersuchung konzentriert sich ausschließlich auf die Benutzerfreundlichkeit von Low-Code-Plattformen. 
Das Paper wurde 2022 veröffentlicht und bezieht sich auf aktuelle Konzepte und Entwicklungen in der Low-Code-Entwicklung. 
Es wird nicht explizit zwischen kommerziellen und Open-Source-Lösungen unterschieden, 
sondern allgemein die Usability von Low-Code-Plattformen untersucht.

\paragraph{Low-Code Development Platforms - A Literature Review}

Das Paper von Niculin Prinz und Melanie Huber (2021)~\cite{Prinz_2021} bietet einen umfassenden Literaturüberblick über Low-Code-Entwicklungsplattformen. 
Es untersucht die verschiedenen Aspekte und Eigenschaften dieser Plattformen, einschließlich ihrer Vorteile, Herausforderungen 
und Anwendungsbereiche. Die Autoren analysieren eine breite Palette von Studien, um die wesentlichen Merkmale und Trends 
in der Low-Code-Entwicklung zu identifizieren.

Das Paper identifiziert und bewertet mehrere Low-Code-Entwicklungsplattformen, darunter auch spezifische Plattformen wie 
OutSystems, Mendix und Appian. Es werden die Vorteile dieser Plattformen, wie z.B. die erhöhte Entwicklungsgeschwindigkeit und 
die Reduzierung von Kosten, sowie die Herausforderungen, wie die Abhängigkeit von proprietären Lösungen und die begrenzte 
Anpassungsfähigkeit, diskutiert. Möglichkeiten der Anwendung im Quantencomputing werden in diesem Paper nicht behandelt. 
Der Fokus liegt auf der allgemeinen Analyse und Bewertung von Low-Code-Plattformen im Kontext 
klassischer Softwareentwicklungsprozesse. Das Paper wurde 2021 veröffentlicht und 
bezieht sich auf aktuelle Konzepte und Entwicklungen in der Low-Code-Entwicklung. Es wird sowohl auf kommerzielle als 
auch auf Open-Source-Lösungen eingegangen, wobei ein Schwerpunkt auf der Bewertung der allgemeinen Marktsituation liegt.

\paragraph{AI for Low-Code for AI}

Das Paper von Nikitha Rao, Jason Tsay, Kiran Kate, Vincent Hellendoorn und Martin Hirzel (2024)~\cite{rao2024} untersucht den Einsatz 
von Künstlicher Intelligenz (KI) zur Verbesserung von Low-Code-Plattformen, die wiederum zur Entwicklung von KI-Anwendungen 
verwendet werden. Das Ziel des Papers ist es, die Synergien zwischen Low-Code-Entwicklung und KI zu beleuchten und zu 
zeigen, wie diese Technologien zusammenarbeiten können, um die Anwendungsentwicklung zu beschleunigen und zu vereinfachen.

Das Paper stellt verschiedene Low-Code-Plattformen vor, die speziell für die Entwicklung von KI-Anwendungen optimiert wurden, 
darunter Tools und Frameworks, die die Erstellung und Implementierung von KI-Modellen unterstützen. 
Es werden die Vorteile dieser Ansätze hervorgehoben, wie z.B. die Vereinfachung komplexer Entwicklungsprozesse, 
die Verbesserung der Zugänglichkeit für nicht-technische Nutzer und die Beschleunigung der Entwicklungszyklen. 
Möglichkeiten der Anwendung im Quantencomputing werden in diesem Paper nicht diskutiert. 
Der Fokus liegt auf der Integration von KI in Low-Code-Plattformen und deren Anwendung in der klassischen Softwareentwicklung. 
Das Paper wurde 2024 veröffentlicht und behandelt die neuesten Entwicklungen und Trends in diesem Bereich. 
Es wird sowohl auf kommerzielle als auch auf Open-Source-Lösungen eingegangen, wobei die 
Vorteile und Herausforderungen der Integration von KI in Low-Code-Entwicklungsplattformen analysiert werden.

\paragraph{Challenges \& Opportunities in Low-Code Testing}

Die Publikation von Faezeh Khorram, Jean-Marie Mottu und Gerson Sunyé (2020)~\cite{Khorram_2020} untersucht die Herausforderungen und Chancen 
im Zusammenhang mit dem Testen von Low-Code-Anwendungen. Sie bietet eine detaillierte Analyse der spezifischen Anforderungen 
und Hindernisse, die bei der Qualitätssicherung von Anwendungen auftreten, die mit Low-Code-Plattformen entwickelt wurden.

In der Publikation werden mehrere Low-Code-Entwicklungsplattformen untersucht, darunter kommerzielle Tools wie OutSystems, 
Mendix und Appian. Ein besonderer Fokus liegt auf den neuen Konzepten und Charakteristika, die Low-Code-Plattformen in 
den Entwicklungsprozess einführen und die damit verbundenen Auswirkungen auf Teststrategien. Möglichkeiten der Anwendung 
im Quantencomputing werden in dieser Publikation nicht behandelt. Der Schwerpunkt liegt auf den Herausforderungen 
des Low-Code-Testings, einschließlich der Rolle von "Citizen Developers" im Testprozess, der Notwendigkeit hochgradiger 
Testautomatisierung und der speziellen Anforderungen an Cloud-Testing. Die Publikation wurde 2020 veröffentlicht und 
behandelt aktuelle Entwicklungen und Konzepte im Bereich des Low-Code-Testings. Es wird sowohl auf kommerzielle als 
auch auf Open-Source-Ansätze eingegangen, wobei die verschiedenen Herausforderungen und Chancen im Zusammenhang mit 
dem Testen von Low-Code-Anwendungen detailliert analysiert werden.

\paragraph{Quantumoonlight: A Low-Code Platform to Experiment with Quantum Machine Learning}

Die Publikation von Francesco Amato und Matteo Cicalese (2023)~\cite{Amato_2023} stellt die Low-Code-Plattform Quantumoonlight vor, 
die speziell für Experimente mit Quantum Machine Learning (QML) entwickelt wurde. Diese Plattform zielt darauf ab, 
die Komplexität der Entwicklung von QML-Anwendungen zu reduzieren und sowohl Forschern als auch Entwicklern eine 
benutzerfreundliche Umgebung zu bieten, in der sie ihre Algorithmen testen und validieren können.

Quantumoonlight bietet eine Reihe von Tools und Funktionen, die es ermöglichen, Quantenalgorithmen zu modellieren und in 
einer intuitiven, visuellen Umgebung zu implementieren. Die Plattform erleichtert den Einstieg in das Quantencomputing, 
indem sie die technischen Hürden senkt und es Nutzern ermöglicht, sich auf die Entwicklung und Optimierung von Algorithmen zu konzentrieren. 

Die Publikation hebt hervor, dass Quantumoonlight sowohl kommerziell als auch als Open-Source-Lösung verfügbar ist, 
was die Zugänglichkeit und Anpassungsfähigkeit der Plattform erhöht. Durch die Bereitstellung von Open-Source-Komponenten 
können Nutzer die Plattform an ihre spezifischen Bedürfnisse anpassen und weiterentwickeln. Möglichkeiten der 
Anwendung im Quantencomputing werden explizit behandelt, und die Plattform wird als potenzielles Werkzeug 
für die breitere Akzeptanz und Entwicklung von Quantum Machine Learning positioniert. Die Publikation wurde 2023 
veröffentlicht und reflektiert die neuesten Trends und Entwicklungen im Bereich der Low-Code-Entwicklung und des Quantencomputings.

\paragraph{Towards a Quantum Software Modeling Language}

Die Publikation von Carlos A. Pérez-Delgado und Héctor G. Pérez-González (2020)~\cite{Perez-Delgado_2020} untersucht die Notwendigkeit und die 
Prinzipien einer Modellierungssprache für Quantensoftware. Ziel der Arbeit ist es, die Entwicklung von 
Quantenanwendungen durch die Einführung formaler Modellierungstechniken zu vereinfachen, die in der 
klassischen Softwareentwicklung bereits weit verbreitet sind.

Die Autoren schlagen eine minimale Erweiterung der bekannten Unified Modeling Language (UML) vor, um sie für die 
Modellierung von Quantensoftware effektiv nutzbar zu machen. Diese Erweiterungen sind modular und können unabhängig 
von der restlichen UML-Struktur angewendet werden, was ihre Integration in andere Modellierungssprachen 
oder die Entwicklung einer völlig neuen Sprache ermöglicht. Die Publikation hebt die Notwendigkeit einer solchen Sprache 
hervor, um die Abstraktionsebene zu erhöhen und die Komplexität der Quantenprogrammierung zu reduzieren.

Es werden keine spezifischen Low-Code-Tools oder Plattformen erwähnt, jedoch wird die Relevanz der 
formalen Modellierung für die Low-Code-Entwicklung von Quantenanwendungen betont. 
Die Konzepte und Tools, die in der Publikation diskutiert werden, sind aktuell und spiegeln den Stand der Forschung im Jahr 2020 wider. 
Möglichkeiten der Anwendung im Quantencomputing werden detailliert beschrieben, wobei der Fokus auf der Modellierung 
und Abstraktion liegt. Die Publikation behandelt keine spezifischen kommerziellen oder Open-Source-Lösungen, sondern 
konzentriert sich auf die theoretischen Grundlagen und die praktische Umsetzung einer Modellierungssprache für Quantensoftware.

\paragraph{Model-Driven Quantum Federated Learning (QFL)}

Die Publikation von Armin Moin, Atta Badii und Moharram Challenger (2023)~\cite{Moin_2023} untersucht die Anwendung von 
Model-Driven Engineering (MDE) im Kontext von Quantum Federated Learning (QFL). Ziel der Arbeit ist es, die Entwicklung und 
Implementierung von Quantum Machine Learning (QML) Anwendungen durch den Einsatz von modellgetriebenen Ansätzen zu vereinfachen und zu beschleunigen.

Die Autoren schlagen vor, bestehende MDE-Tools für maschinelles Lernen, wie MontiAnna und ML-Quadrat, zu erweitern, um QFL 
zu unterstützen. Diese Erweiterungen sollen es Entwicklern ermöglichen, QFL-Modelle auf einer höheren 
Abstraktionsebene zu entwerfen und zu implementieren, ohne tiefgehendes Fachwissen über Quantencomputing zu benötigen. 
Durch die Bereitstellung von Abstraktionsschichten und automatisierten Modell-zu-Code-Transformationen wird die Komplexität 
der Entwicklung von QFL-Anwendungen reduziert.

Die Publikation hebt hervor, dass die vorgeschlagenen Ansätze sowohl für kommerzielle als auch für 
Open-Source-Lösungen relevant sind. Sie betont die Notwendigkeit offener Plattformen, um die Zugänglichkeit und 
Anpassungsfähigkeit der entwickelten Tools zu gewährleisten. Die Möglichkeiten der Anwendung im Quantencomputing werden 
ausführlich diskutiert, wobei der Fokus auf der Integration von MDE und QML liegt. Die in der Publikation 
behandelten Konzepte und Tools sind aktuell und reflektieren den Stand der Forschung im Jahr 2023. 
Die Autoren argumentieren, dass die Verwendung von MDE-Ansätzen die Entwicklung von QFL-Anwendungen 
erheblich erleichtern und beschleunigen kann, was insbesondere für die Forschung und 
Entwicklung in diesem schnell wachsenden Bereich von großer Bedeutung ist.

\paragraph{Towards Model-Driven Quantum Software Engineering}

Die Publikation von Felix Gemeinhardt, Antonio Garmendia und Manuel Wimmer (2021)~\cite{gemeinhardt2021towards} beleuchtet die Anwendung von 
Model-Driven Engineering (MDE) im Bereich der Quantensoftwareentwicklung. Die Autoren argumentieren, dass MDE-Prinzipien, 
die sich in der klassischen Softwareentwicklung bewährt haben, auch auf die Entwicklung von Quantensoftware angewendet 
werden können, um die Komplexität zu reduzieren und die Effizienz zu steigern.

In der Arbeit wird ein spezifischer Forschungsansatz vorgestellt, der die Nutzung von Domain-Specific Modeling Languages (DSMLs) und 
generativen Techniken wie der Codegenerierung für die Entwicklung von Quantensoftware untersucht. Die Autoren 
präsentieren ein Demonstrationsszenario, in dem MDE-Techniken verwendet werden, um Modelle für die Analyse 
sozialer Netzwerke auf Quantencomputern zu erstellen. Dieses Szenario dient als Beweis für die Machbarkeit 
und den Nutzen von MDE in der Quantensoftwareentwicklung.

Die Publikation ist aktuell und reflektiert die neuesten Trends und Entwicklungen im Jahr 2021. Möglichkeiten der Anwendung 
im Quantencomputing werden ausführlich behandelt, wobei der Fokus auf der Integration von MDE-Techniken liegt. 
Die Autoren betonen die Relevanz von Open-Source-Lösungen, um die Zugänglichkeit und Anpassungsfähigkeit der 
entwickelten Tools zu gewährleisten. Durch die Bereitstellung offener Plattformen können Forscher und Entwickler 
die vorgestellten Konzepte weiterentwickeln und anpassen.

Insgesamt stellt die Arbeit eine wichtige Grundlage für die Erforschung und Implementierung von MDE-Prinzipien 
in der Quantensoftwareentwicklung dar. Sie zeigt auf, wie die Anwendung dieser Prinzipien die Entwicklung von 
Quantensoftware beschleunigen und die Eintrittsbarrieren für neue Entwickler senken kann.

\paragraph{A Model-Driven Framework for Composition-Based Quantum Circuit Design}

Die Publikation von Felix Gemeinhardt, Antonio Garmendia, Manuel Wimmer und R. Wille (2018)~\cite{Gemeinhardt_2018}
stellt ein modellgetriebenes Framework zur Gestaltung von zusammensetzungsbasierten Quantenschaltkreisen vor. Ziel der Arbeit 
ist es, die Entwicklung von Quantenschaltungen durch die Anwendung von Model-Driven Engineering (MDE) zu erleichtern und zu beschleunigen.

Die Autoren präsentieren ein Framework, das es ermöglicht, Quantenschaltkreise auf einer höheren Abstraktionsebene zu entwerfen. 
Dies geschieht durch die Nutzung von Domain-Specific Modeling Languages (DSMLs) und generativen Techniken, die 
es Entwicklern erlauben, komplexe Quantenoperationen durch die Komposition einfacher Bausteine zu erstellen. 
Durch die Verwendung von Modellen können Quantenschaltungen visuell entworfen und automatisch in ausführbaren Quantenprogrammcode umgewandelt werden.

Die Publikation diskutiert detailliert die Funktionsweise des Frameworks und zeigt auf, wie verschiedene 
Quantenlogikgatter und -operationen modelliert und zusammengesetzt werden können. Ein besonderer Schwerpunkt liegt auf 
der Integration von Open-Source-Tools, um die Zugänglichkeit und Anpassungsfähigkeit des Frameworks zu gewährleisten. 
Die Autoren betonen, dass die Verwendung von Open-Source-Plattformen entscheidend ist, um die Weiterentwicklung und Verbreitung des Frameworks zu fördern.

Die in der Publikation behandelten Konzepte und Tools sind aktuell und spiegeln den Stand der Forschung im Jahr 2018 wider. 
Die Arbeit zeigt, wie MDE-Techniken die Entwicklung von Quantenschaltungen vereinfachen und beschleunigen können, indem 
sie die Komplexität reduzieren und die Effizienz steigern. Die vorgestellten Ansätze sind sowohl für 
kommerzielle als auch für Open-Source-Anwendungen relevant und bieten eine solide Grundlage für zukünftige Entwicklungen in der Quantenschaltungsentwicklung.

\paragraph{A Reference Architecture for Quantum Computing as a Service}

Die Publikation von Aakash Ahmad, Ahmed B. Altamimi und Jamal M. Aqib (2023)~\cite{Ahmad_2023} stellt eine Referenzarchitektur 
für Quantum Computing as a Service (QCaaS) vor. Ziel der Arbeit ist es, ein Framework zu entwickeln, das es ermöglicht, 
Quantencomputing-Dienste über Cloud-basierte Plattformen anzubieten und zu nutzen.

Die Autoren präsentieren eine detaillierte Architektur, die verschiedene Komponenten und Schichten beschreibt, 
die erforderlich sind, um QCaaS zu implementieren. Dazu gehören unter anderem die Verwaltung der 
Quantenressourcen, die Integration von Quanten- und klassischen Berechnungen, sowie die Bereitstellung von 
Entwicklungswerkzeugen für Quantenalgorithmen. Die Architektur soll es Entwicklern und Nutzern 
ermöglichen, Quantencomputing-Dienste zu nutzen, ohne tiefgehende Kenntnisse über die zugrunde liegende Quantenhardware zu benötigen.

Ein besonderes Augenmerk liegt auf der Integration von Open-Source-Tools und -Plattformen, um die 
Zugänglichkeit und Anpassungsfähigkeit der angebotenen Dienste zu gewährleisten. Die Autoren betonen, 
dass Open-Source-Lösungen entscheidend sind, um die Weiterentwicklung und Verbreitung von QCaaS zu fördern und die 
Barrieren für den Zugang zu Quantencomputing-Technologien zu senken.

Die Publikation diskutiert auch die technischen Herausforderungen, die mit der Bereitstellung von QCaaS verbunden sind, 
wie z.B. die Skalierbarkeit und Performance von Quantencomputing-Diensten. Die Autoren bieten Lösungsansätze für diese 
Herausforderungen und zeigen auf, wie die vorgestellte Referenzarchitektur implementiert und genutzt werden kann.

Insgesamt bietet die Arbeit eine umfassende und aktuelle Darstellung der Konzepte und Technologien, die 
für die Implementierung von QCaaS erforderlich sind. Die vorgestellten Ansätze sind sowohl für 
kommerzielle als auch für Open-Source-Anwendungen relevant und bieten eine solide Grundlage für die 
Entwicklung und Bereitstellung von Quantencomputing-Diensten über Cloud-Plattformen.

\paragraph{Design of classical-quantum systems with UML}

Die Publikation von Ricardo Pérez-Castillo und Mario Piattini (2022)~\cite{Perez-Castillo_2022} untersucht die Verwendung von Unified Modeling 
Language (UML) für das Design von hybriden Systemen, die sowohl klassische als auch Quantenkomponenten enthalten. 
Ziel der Arbeit ist es, eine methodische Grundlage zu schaffen, die es ermöglicht, die besonderen Anforderungen und 
Herausforderungen bei der Entwicklung solcher hybriden Systeme zu adressieren.

Die Autoren präsentieren eine UML-Erweiterung, die spezifisch für die Modellierung von Quantenalgorithmen und -systemen 
entwickelt wurde. Diese Erweiterung umfasst neue Stereotypen und Modellelemente, die es ermöglichen, Quantenoperationen 
und -logikgatter innerhalb von UML-Diagrammen darzustellen. Durch diese Integration können Entwickler sowohl klassische 
als auch Quantenkomponenten in einem einheitlichen Modell entwerfen und analysieren.

Ein besonderer Schwerpunkt liegt auf der Unterstützung von Open-Source-Ansätzen, um die Nutzung und Weiterentwicklung der 
vorgestellten Methoden und Tools zu fördern. Die Autoren betonen, dass Open-Source-Plattformen entscheidend sind, um die 
Anpassungsfähigkeit und Verbreitung der entwickelten UML-Erweiterungen zu gewährleisten.

Die Publikation diskutiert auch die Herausforderungen, die mit der Modellierung von Quantenkomponenten verbunden sind, 
wie z.B. die Darstellung von Superposition und Verschränkung in UML-Diagrammen. Die Autoren bieten Lösungen für diese 
Herausforderungen und zeigen auf, wie die vorgestellten Modellelemente in der Praxis angewendet werden können.

Insgesamt bietet die Arbeit eine umfassende und aktuelle Darstellung der Konzepte und Technologien, die für das Design 
von hybriden klassischen und Quanten-Systemen erforderlich sind. Die vorgestellten Ansätze sind sowohl für kommerzielle 
als auch für Open-Source-Anwendungen relevant und bieten eine solide Grundlage für die Weiterentwicklung von Modellierungstechniken 
im Bereich des Quantencomputings.

\paragraph{Integrating Quantum Computing into Workflow Modeling and Execution}

Die Publikation von Benjamin Weder, Uwe Breitenbücher, Frank Leymann und Karoline Wild (2020)~\cite{Weder_2020} befasst sich mit der 
Integration von Quantencomputing in die Modellierung und Ausführung von Workflows. Ziel der Arbeit ist es, eine 
methodische Grundlage zu schaffen, die es ermöglicht, Quantenoperationen nahtlos in bestehende Workflow-Modelle zu 
integrieren und so die Vorteile des Quantencomputings in verschiedenen Anwendungsbereichen nutzbar zu machen.

Die Autoren präsentieren eine Erweiterung für imperative Workflow-Sprachen, die es ermöglicht, Quantenoperationen 
zu modellieren und in Workflows zu integrieren. Diese Erweiterung beinhaltet spezifische Konstrukte und Modellelemente, 
die Quantenlogikgatter und -operationen darstellen. Durch diese Integration können Entwickler komplexe Workflows entwerfen, 
die sowohl klassische als auch Quantenkomponenten enthalten, und diese in einer einheitlichen Umgebung ausführen.

Ein besonderer Schwerpunkt der Arbeit liegt auf der Unterstützung von Open-Source-Ansätzen, um die Nutzung und 
Weiterentwicklung der vorgestellten Methoden und Tools zu fördern. Die Autoren betonen, dass Open-Source-Plattformen 
entscheidend sind, um die Anpassungsfähigkeit und Verbreitung der entwickelten Erweiterungen zu gewährleisten.

Die Publikation diskutiert auch die technischen Herausforderungen, die mit der Integration von Quantenoperationen in 
Workflows verbunden sind, wie z.B. die Handhabung von Fehlern und die Sicherstellung der Skalierbarkeit und Performance. 
Die Autoren bieten Lösungen für diese Herausforderungen und zeigen auf, wie die vorgestellten Modellelemente und Methoden 
in der Praxis angewendet werden können.

Insgesamt bietet die Arbeit eine umfassende und aktuelle Darstellung der Konzepte und Technologien, die für die Integration 
von Quantencomputing in Workflow-Modelle erforderlich sind. Die vorgestellten Ansätze sind sowohl für kommerzielle als auch 
für Open-Source-Anwendungen relevant und bieten eine solide Grundlage für die Weiterentwicklung von Workflow-Management-Systemen 
im Bereich des Quantencomputings.

\paragraph{Towards Quantum-algorithms-as-a-service}

Die Publikation von Manuel De Stefano, Dario Di Nucci, Fabio Palomba, Davide Taibi und Andrea De Lucia (2022)~\cite{Stefano_2022} untersucht 
das Konzept der Bereitstellung von Quantenalgorithmen als Dienstleistung, auch bekannt als Quantum-Algorithms-as-a-Service (QAaaS). 
Ziel der Arbeit ist es, ein Entwicklungsmodell zu schaffen, das Entwicklern die Abstraktion der Quantenkomponenten von der 
Software, die sie erstellen, ermöglicht und somit die Integration von Quantencomputing in bestehende Systeme erleichtert.

Die Autoren präsentieren ein Modell, das Software-as-a-Service (SaaS) und Function-as-a-Service (FaaS) kombiniert, um 
die Ausführung von Quantenalgorithmen über verschiedene Quanten-Cloud-Anbieter hinweg zu unterstützen. Dies ermöglicht es 
Entwicklern, Quantenalgorithmen zu nutzen, ohne sich um die spezifischen Details der zugrunde liegenden Quantenhardware kümmern zu müssen. 

Ein besonderer Schwerpunkt der Arbeit liegt auf der Nutzung von Open-Source-Ansätzen, um die Verfügbarkeit und 
Anpassungsfähigkeit der vorgestellten Methoden und Tools zu verbessern. Die Autoren betonen, dass Open-Source-Plattformen 
entscheidend sind, um die Zusammenarbeit und Innovation in der Quantencomputing-Community zu fördern.

Die Publikation diskutiert auch die Herausforderungen, die mit der Bereitstellung von Quantenalgorithmen als 
Dienstleistung verbunden sind, wie z.B. die Sicherstellung der Interoperabilität zwischen verschiedenen Quanten-Cloud-Anbietern 
und die Handhabung der Performance- und Skalierbarkeitsanforderungen. Die Autoren bieten Lösungen für diese Herausforderungen 
und zeigen auf, wie die vorgestellten Modellelemente und Methoden in der Praxis angewendet werden können.

Insgesamt bietet die Arbeit eine umfassende und aktuelle Darstellung der Konzepte und Technologien, die für die 
Bereitstellung von Quantenalgorithmen als Dienstleistung erforderlich sind. Die vorgestellten Ansätze sind sowohl 
für kommerzielle als auch für Open-Source-Anwendungen relevant und bieten eine solide Grundlage für die Weiterentwicklung 
von Quantencomputing-Dienstleistungen.

\paragraph{Generation of Classical-Quantum Code from UML models}

Die Publikation von Ricardo Pérez-Castillo, Luis Jiménez-Navajas, Iván Cantalejo und M. Piattini (2023)~\cite{Perez-Castillo_2023} 
befasst sich mit der Generierung von klassischem und Quanten-Code aus UML-Modellen. Ziel der Arbeit ist es, 
eine Methode zu entwickeln, die es ermöglicht, aus UML-Designmodellen sowohl klassischen als auch Quanten-Code 
zu generieren und somit die Entwicklung hybrider Software-Systeme zu unterstützen.

Die Autoren präsentieren eine Technik zur Erweiterung von UML-Modellen, die es ermöglicht, Quantenkomponenten 
und -operationen zu modellieren. Diese Erweiterungen werden in einem Metamodell spezifiziert, das die Integration 
von klassischen und Quanten-Elementen in ein einheitliches Modell ermöglicht. Die vorgeschlagene Methode verwendet 
eine Modell-zu-Text-Transformation, um aus den UML-Modellen automatisch Code in Python und Qiskit zu generieren.

Ein besonderer Schwerpunkt der Arbeit liegt auf der Nutzung von Open-Source-Ansätzen, um die Verfügbarkeit und 
Weiterentwicklung der vorgestellten Methoden und Tools zu fördern. Die Autoren betonen, dass Open-Source-Plattformen 
entscheidend sind, um die Akzeptanz und Anpassungsfähigkeit der entwickelten Techniken in der Praxis zu gewährleisten.

Die Publikation diskutiert auch die technischen Herausforderungen, die mit der Generierung von klassischem und 
Quanten-Code aus UML-Modellen verbunden sind, wie z.B. die Komplexität der Modelltransformation und die Sicherstellung 
der Korrektheit und Effizienz des generierten Codes. Die Autoren bieten Lösungen für diese Herausforderungen und 
zeigen auf, wie die vorgestellten Techniken in der Praxis angewendet werden können.

Insgesamt bietet die Arbeit eine umfassende und aktuelle Darstellung der Konzepte und Technologien, die für die 
Generierung von klassischem und Quanten-Code aus UML-Modellen erforderlich sind. Die vorgestellten Ansätze sind 
sowohl für kommerzielle als auch für Open-Source-Anwendungen relevant und bieten eine solide Grundlage für die 
Weiterentwicklung von Modellierungstechniken im Bereich des Quantencomputings.

\paragraph{A Graph-Based Approach for Modelling Quantum Circuits}

Die Publikation von Diego Alonso, Pedro Sánchez und Bárbara Álvarez (2023)~\cite{alonso2023graph} stellt einen graphbasierten Ansatz zur 
Modellierung von Quantenschaltungen vor. Ziel der Arbeit ist es, ein einheitliches Metamodell für die Modellierung 
von Quantenschaltungen zu entwickeln, das als Grundlage für die automatische Codegenerierung und Integration mit 
anderen Werkzeugen dient.

Die Autoren präsentieren eine umfassende Analyse der bestehenden Ansätze zur Modellierung von Quantenschaltungen 
und identifizieren die wichtigsten Herausforderungen und Lücken. Basierend auf dieser Analyse wird ein neues Metamodell 
vorgeschlagen, das die graphbasierte Darstellung von Quantenschaltungen ermöglicht. Dieses Metamodell umfasst 
verschiedene Strategien zur Modellierung und Transformation von Quantenschaltungen und stellt sicher, dass die 
Modelle sowohl für die theoretische Analyse als auch für praktische Anwendungen geeignet sind.

Ein besonderer Schwerpunkt der Arbeit liegt auf der Nutzung von Open-Source-Ansätzen, um die Verfügbarkeit und 
Weiterentwicklung der vorgestellten Methoden und Tools zu fördern. Die Autoren betonen, dass Open-Source-Plattformen 
entscheidend sind, um die Zusammenarbeit und Innovation in der Quantencomputing-Community zu unterstützen.

Die Publikation diskutiert auch die technischen Herausforderungen, die mit der graphbasierten Modellierung von 
Quantenschaltungen verbunden sind, wie z.B. die Komplexität der Modelltransformation und die Sicherstellung der 
Korrektheit und Effizienz des generierten Codes. Die Autoren bieten Lösungen für diese Herausforderungen und zeigen 
auf, wie die vorgestellten Techniken in der Praxis angewendet werden können.

Insgesamt bietet die Arbeit eine umfassende und aktuelle Darstellung der Konzepte und Technologien, die für die 
graphbasierte Modellierung von Quantenschaltungen erforderlich sind. Die vorgestellten Ansätze sind sowohl für 
kommerzielle als auch für Open-Source-Anwendungen relevant und bieten eine solide Grundlage für die Weiterentwicklung 
von Modellierungstechniken im Bereich des Quantencomputings.

\paragraph{MDE4QAI: Towards Model-Driven Engineering for Quantum Artificial Intelligence}

Die Publikation von Armin Moin, Moharram Challenger, Atta Badii und Stephan Günnemann (2021)~\cite{moin2021mde4qai} untersucht die Anwendung 
von Model-Driven Engineering (MDE) im Kontext der Quantenkünstlichen Intelligenz (Quantum Artificial Intelligence, QAI). 
Ziel der Arbeit ist es, die Potenziale von MDE für die Entwicklung von Quanten- und hybrid-quantum-klassischen Anwendungen aufzuzeigen.

Die Autoren argumentieren, dass MDE als Enabler und Facilitator für Quantum AI fungieren kann, insbesondere durch 
die Bereitstellung von automatisierten Code-Generierung, Modellprüfung und -validierung sowie Modell-zu-Modell-Transformationen. 
Diese Techniken können nicht nur in den frühen Designphasen, sondern auch zur Laufzeit angewendet werden, um die Entwicklung von QAI-Anwendungen zu erleichtern.

Die Publikation hebt hervor, dass die Integration von MDE in die Entwicklung von QAI-Systemen erhebliche Vorteile 
bietet, darunter die Abstraktion der Komplexität, die Erhöhung der Wiederverwendbarkeit von Modellen und die 
Verbesserung der Effizienz der Entwicklungsprozesse. Die Autoren präsentieren eine Vision für die Verwendung von 
MDE in der Quantenkünstlichen Intelligenz und diskutieren verschiedene Anwendungsfälle und Herausforderungen.

Ein besonderer Schwerpunkt der Arbeit liegt auf der Notwendigkeit, bestehende MDE-Tools und -Methoden zu erweitern, 
um die spezifischen Anforderungen der Quantencomputing- und AI-Entwicklung zu erfüllen. Die Autoren schlagen vor, dass 
Open-Source-Ansätze eine Schlüsselrolle spielen können, um die Verfügbarkeit und Weiterentwicklung der vorgestellten 
Methoden und Tools zu fördern.

Insgesamt bietet die Arbeit eine umfassende und aktuelle Darstellung der Potenziale und Herausforderungen von MDE im 
Kontext der Quantenkünstlichen Intelligenz. Die vorgestellten Ansätze sind sowohl für kommerzielle als auch für 
Open-Source-Anwendungen relevant und bieten eine solide Grundlage für die Weiterentwicklung von Modellierungstechniken 
im Bereich des Quantencomputings.

\paragraph{Model-Driven Engineering for Quantum Programming: A Case Study on Ground State Energy Calculation}

Die Publikation von Furkan Polat, Hasan Tuncer, Armin Moin und Moharram Challenger (2024)~\cite{Polat_2024} untersucht die Anwendung 
von Model-Driven Engineering (MDE) zur Programmierung von Quantencomputern, mit einem speziellen Fokus auf die 
Berechnung der Grundzustandsenergie. Ziel der Arbeit ist es, MDE-Techniken zu nutzen, um die Entwicklung und 
Implementierung von Quantenalgorithmen zu erleichtern.

Die Autoren präsentieren eine Fallstudie, in der sie die Prinzipien von MDE auf die Berechnung der Grundzustandsenergie 
anwenden. Dies umfasst die Nutzung von Modellen zur Abstraktion der Komplexität und die automatische Generierung von 
Code für verschiedene Quantencomputing-Plattformen. Der Ansatz ermöglicht es, die Entwicklungszeit zu verkürzen und 
die Fehleranfälligkeit zu reduzieren.

Ein besonderer Schwerpunkt der Arbeit liegt auf der Integration von Gate-basiertem Quantencomputing und Quantum 
Annealing. Die Autoren entwickeln eine Methode zur Abbildung von Programmen zwischen diesen beiden Paradigmen, was 
zu einer automatischen Transformation der Programme führt. Dies wird am Beispiel des Variational Quantum Eigensolver 
Algorithm und des Quantum Annealing Ising Modells demonstriert.

Die Publikation betont die Bedeutung von Open-Source-Ansätzen zur Förderung der Zusammenarbeit und Innovation in der 
Quantencomputing-Community. Die Autoren argumentieren, dass Open-Source-Tools und -Plattformen entscheidend sind, um 
die Verfügbarkeit und Weiterentwicklung der vorgestellten Methoden und Techniken zu gewährleisten.

Insgesamt bietet die Arbeit eine umfassende und aktuelle Darstellung der Anwendung von MDE im Quantencomputing, mit 
besonderem Fokus auf die Berechnung der Grundzustandsenergie. Die vorgestellten Ansätze sind sowohl für kommerzielle 
als auch für Open-Source-Anwendungen relevant und bieten eine solide Grundlage für die Weiterentwicklung von 
Modellierungstechniken im Bereich des Quantencomputings.

\paragraph{Kdm to uml model transformation for quantum software modernization}

Die Publikation von Luis Jiménez-Navajas, Ricardo Pérez-Castillo und M. Piattini (2021)~\cite{jimenez2021kdm} untersucht die Transformation 
von KDM (Knowledge Discovery Metamodel) zu UML (Unified Modeling Language) Modellen zur Modernisierung von 
Quantensoftware. Ziel der Arbeit ist es, Methoden und Techniken zu entwickeln, die eine nahtlose Integration 
von Quanten- und klassischen Softwarekomponenten ermöglichen.

Die Autoren stellen einen Ansatz vor, bei dem KDM-Modelle, die häufig zur Dokumentation und Analyse bestehender 
Softwaresysteme verwendet werden, in UML-Modelle transformiert werden. Diese Transformation erleichtert die 
Analyse, das Design und die Weiterentwicklung von Quantensoftware, indem sie eine einheitliche Modellierungssprache 
bereitstellt, die von vielen bestehenden Tools und Methoden unterstützt wird.

Ein besonderer Schwerpunkt der Arbeit liegt auf der Verwendung von MDE-Techniken, um die Transformation und 
Integration zu automatisieren. Die Autoren demonstrieren die Anwendbarkeit ihres Ansatzes anhand von Fallstudien, 
die die Effizienz und Effektivität der vorgeschlagenen Methoden zur Modelltransformation belegen. Dies umfasst 
die Modellierung von hybriden Systemen, die sowohl klassische als auch Quantenkomponenten enthalten.

Die Publikation hebt hervor, dass die Verwendung von UML zur Modellierung von Quantensoftware mehrere Vorteile 
bietet, darunter die Verbesserung der Verständlichkeit und Wartbarkeit der Software, sowie die Förderung der 
Wiederverwendbarkeit von Modellen und Komponenten. Die Autoren argumentieren, dass Open-Source-Ansätze entscheidend 
sind, um die Verfügbarkeit und Weiterentwicklung der vorgestellten Methoden und Tools zu gewährleisten.

Insgesamt bietet die Arbeit eine umfassende und aktuelle Darstellung der Techniken und Methoden zur Modernisierung 
von Quantensoftware durch Modelltransformation. Die vorgestellten Ansätze sind sowohl für kommerzielle als auch 
für Open-Source-Anwendungen relevant und bieten eine solide Grundlage für die Weiterentwicklung von 
Modellierungstechniken im Bereich des Quantencomputings.

\paragraph{Modeling Quantum programs: challenges, initial results, and research directions}

Die Publikation von Shaukat Ali und Tao Yue (2020)~\cite{ali2020modeling} untersucht die Herausforderungen, ersten Ergebnisse und 
zukünftigen Forschungsschwerpunkte im Bereich der Modellierung von Quantenprogrammen. Ziel der Arbeit ist es, 
die Besonderheiten und Anforderungen der Quantenprogrammierung zu analysieren und geeignete Modellierungstechniken 
zu entwickeln, die die Komplexität der Quantencomputing-Programmierung reduzieren können.

Die Autoren identifizieren mehrere zentrale Herausforderungen in der Quantenprogrammierung, darunter die 
Notwendigkeit, die Prinzipien der Quantenmechanik wie Superposition und Verschränkung zu verstehen und anzuwenden. 
Diese Konzepte unterscheiden sich grundlegend von denen der klassischen Programmierung und erfordern daher neue 
Ansätze für die Modellierung und Entwicklung von Software.

Die Publikation präsentiert erste Ergebnisse aus der Entwicklung von Modellierungstechniken, die speziell auf 
Quantenprogramme zugeschnitten sind. Ein Schwerpunkt liegt dabei auf der Entwicklung abstrakter Modellierungssprachen, 
die es Entwicklern ermöglichen, Quantenprogramme auf einer höheren Abstraktionsebene zu entwerfen. Dies umfasst die 
Verwendung von Modelltransformationen, um die automatisierte Generierung von Quantenprogrammen zu unterstützen.

Ein wichtiger Aspekt der Arbeit ist die Diskussion der aktuellen Forschungsergebnisse und der daraus resultierenden 
zukünftigen Forschungsrichtungen. Die Autoren betonen die Notwendigkeit weiterer Untersuchungen, um robuste und 
skalierbare Modellierungstechniken zu entwickeln, die den Anforderungen der Quantenprogrammierung gerecht werden. 
Sie schlagen vor, dass Open-Source-Ansätze und Kollaborationen zwischen Forschungseinrichtungen und Industrie 
wesentlich sind, um die Verbreitung und Weiterentwicklung der vorgestellten Methoden zu fördern.

Insgesamt bietet die Arbeit eine umfassende Analyse der Herausforderungen und Möglichkeiten in der Modellierung 
von Quantenprogrammen und zeigt auf, wie MDE-Techniken zur Bewältigung dieser Herausforderungen beitragen können. 
Die vorgestellten Ansätze sind sowohl für kommerzielle als auch für Open-Source-Anwendungen relevant und bieten 
eine solide Grundlage für die Weiterentwicklung der Quantenprogrammierung.

\paragraph{Transforming Quantum Programs in Kdm to Quantum Design Models in Uml}

Die Publikation von Luis Jiménez-Navajas, Ricardo Pérez-Castillo und Mario Piattini (2022)~\cite{Jimenez-Navajas_2022} untersucht die 
Transformation von Quantenprogrammen, die im Knowledge Discovery Metamodel (KDM) dargestellt sind, in 
UML (Unified Modeling Language) Design-Modelle. Ziel dieser Arbeit ist es, Methoden und Techniken zu 
entwickeln, die die Integration von Quanten- und klassischen Softwarekomponenten durch die Nutzung 
standardisierter Modellierungssprachen erleichtern.

Die Autoren stellen einen systematischen Ansatz vor, bei dem KDM-Modelle, die häufig zur Dokumentation und 
Analyse bestehender Softwaresysteme verwendet werden, in UML-Design-Modelle transformiert werden. Dieser Ansatz 
ermöglicht eine einheitliche Darstellung von Quanten- und klassischen Softwareelementen und fördert die 
Wiederverwendbarkeit und Wartbarkeit der Software.

Ein besonderer Fokus der Arbeit liegt auf der Anwendung von MDE-Techniken (Model-Driven Engineering) zur Automatisierung 
der Transformation und Integration von Modellen. Die Autoren demonstrieren die Anwendbarkeit ihres Ansatzes durch 
Fallstudien, die die Effizienz und Effektivität der vorgeschlagenen Methoden zur Modelltransformation belegen. 
Diese Fallstudien umfassen die Modellierung hybrider Systeme, die sowohl klassische als auch Quantenkomponenten enthalten.

Die Publikation hebt hervor, dass die Verwendung von UML zur Modellierung von Quantenprogrammen mehrere Vorteile 
bietet, darunter die Verbesserung der Verständlichkeit und Wartbarkeit der Software sowie die Förderung der 
Wiederverwendbarkeit von Modellen und Komponenten. Die Autoren betonen, dass Open-Source-Ansätze entscheidend sind, um 
die Verfügbarkeit und Weiterentwicklung der vorgestellten Methoden und Tools zu gewährleisten.

Insgesamt bietet die Arbeit eine umfassende und aktuelle Darstellung der Techniken und Methoden zur Transformation von 
Quantenprogrammen in UML-Modelle. Die vorgestellten Ansätze sind sowohl für kommerzielle als auch für Open-Source-Anwendungen 
relevant und bieten eine solide Grundlage für die Weiterentwicklung von Modellierungstechniken im Bereich des Quantencomputings.

\paragraph{Toward a Quantum Software Engineering}

Die Publikation von Mario Piattini, Manuel A. Serrano, Ricardo Pérez-Castillo, Guido Petersen und José Luis Hevia (2021)~\cite{Piattini_2021} 
diskutiert die Notwendigkeit und die Herausforderungen der Entwicklung einer neuen Disziplin: der Quantum Software 
Engineering (QSE). Diese Disziplin zielt darauf ab, die Prinzipien und Praktiken der klassischen Softwaretechnik 
auf die Entwicklung von Quantensoftware zu übertragen, um die Qualität, Wartbarkeit und Wiederverwendbarkeit von 
Quantenanwendungen zu gewährleisten.

Die Autoren betonen, dass Quantencomputing ein völlig neues Paradigma darstellt, das sich erheblich von der 
klassischen Informatik unterscheidet. Quantencomputersysteme basieren auf den Prinzipien der Quantenmechanik 
wie Superposition und Verschränkung, die es ermöglichen, bestimmte Probleme exponentiell schneller zu lösen als 
klassische Computer. Aufgrund dieser fundamentalen Unterschiede erfordert die Entwicklung von Quantensoftware neue 
Modelle, Methoden und Werkzeuge, die speziell auf die Anforderungen des Quantencomputings zugeschnitten sind.

Ein zentraler Aspekt der Arbeit ist die Notwendigkeit einer systematischen Herangehensweise an die 
Quantensoftwareentwicklung. Die Autoren schlagen vor, bewährte Praktiken der klassischen Softwaretechnik, wie 
beispielsweise Modellierung, Design Patterns und automatisierte Tests, in den Kontext des Quantencomputings zu 
übertragen. Dies umfasst auch die Entwicklung neuer Modellierungssprachen und Frameworks, die die Spezifikation 
und Implementierung von Quantenalgorithmen unterstützen.

Darüber hinaus diskutieren die Autoren die Herausforderungen, die mit der Entwicklung von hybriden Systemen verbunden 
sind, die sowohl klassische als auch Quantenkomponenten enthalten. Sie betonen die Bedeutung von Modellierungstechniken, 
die eine nahtlose Integration beider Komponenten ermöglichen, um die Komplexität solcher Systeme zu bewältigen und 
ihre Wartbarkeit zu verbessern.

Die Arbeit hebt auch die Rolle von Open-Source-Initiativen hervor, um die Verfügbarkeit und Weiterentwicklung von Tools 
und Methoden im Bereich der Quantensoftwaretechnik zu fördern. Durch die Bereitstellung frei zugänglicher Ressourcen 
können Forscher und Entwickler weltweit von den Fortschritten in diesem Bereich profitieren und gemeinsam an der Lösung 
der Herausforderungen arbeiten.

Insgesamt bietet die Publikation eine umfassende und aktuelle Darstellung der aufkommenden Disziplin der Quantensoftwaretechnik. 
Die vorgestellten Ansätze und Methoden bilden eine solide Grundlage für die zukünftige Forschung und Entwicklung in diesem 
Bereich und tragen dazu bei, die Potenziale des Quantencomputings voll auszuschöpfen.

\paragraph{Software architecture for quantum computing systems – A systematic review}

Die Publikation von Arif Ali Khan, Aakash Ahmad, Muhammad Waseem, Peng Liang, Mahdi Fahmideh, Tommi Mikkonen und Pekka 
Abrahamsson (2023)~\cite{khan2023software} bietet eine systematische Übersicht über die Softwarearchitektur für Quantencomputersysteme. Ziel 
dieser systematischen Literaturübersicht ist es, die bestehenden architektonischen Ansätze, Modellierungsnotationen, 
Designmuster, Werkzeugunterstützungen und herausfordernden Faktoren für die Architektur von Quantensoftware zu 
identifizieren und zu analysieren.

Die Autoren untersuchen eine Vielzahl von Quellen, um die aktuellen Trends und Entwicklungen im Bereich der Softwarearchitektur 
für Quantencomputing zu erfassen. Dabei werden zentrale Fragestellungen adressiert, wie beispielsweise die Anpassung bestehender 
Architekturprozesse und Modellierungsnotationen für Quantenanwendungen, die Entwicklung neuer Architekturdesignmuster speziell 
für Quantencomputersysteme sowie die Identifikation und Bewertung von Werkzeugen zur Unterstützung der Architekturentwicklung.

Ein zentrales Ergebnis der Untersuchung ist, dass Quantensoftware eine neue Art von softwareintensiven Systemen darstellt, für 
die bestehende Prozesse und Notationen angepasst werden können, um die architektonischen Aktivitäten zu unterstützen und 
Modellierungssprachen für Quantensoftware zu entwickeln. Ein besonderes Augenmerk liegt auf der Abbildung von 
Quantenbits (Qubits) und Quantengattern (Qugates) als architektonische Komponenten und Verbindungen, die die 
Implementierung von Quantensoftware ermöglichen.

Darüber hinaus werden die Herausforderungen bei der Entwicklung und Evolution von Quantensoftwarearchitekturen thematisiert. 
Dazu gehören die Komplexität der Integration von Quanten- und klassischen Komponenten, die Notwendigkeit spezialisierter 
Werkzeuge zur Unterstützung der Modellierung und Analyse von Quantensoftware sowie die Berücksichtigung spezifischer 
Anforderungen und Beschränkungen des Quantencomputings.

Die Publikation hebt zudem die Bedeutung von Tool-Chains hervor, die wiederverwendbares Wissen und menschliche 
Rollen (z.B. Quanten-Domäneningenieure, Quantencode-Entwickler) einbinden, um den architektonischen Prozess zu 
automatisieren und anzupassen. Die Ergebnisse dieser systematischen Literaturübersicht bieten Forschern und Praktikern 
wertvolle Einblicke und dienen als Grundlage für die Entwicklung neuer Hypothesen, die Ableitung von 
Referenzarchitekturen und die Anwendung architekturzentrierter Prinzipien und Praktiken zur Entwicklung der nächsten Generation von Quantensoftware.

Insgesamt liefert diese systematische Literaturübersicht eine umfassende Analyse der aktuellen Forschung und 
Praxis im Bereich der Softwarearchitektur für Quantencomputersysteme und trägt dazu bei, die zukünftige Entwicklung 
und Implementierung dieser Systeme zu unterstützen.

\paragraph{Classical to Quantum Software Migration Journey Begins: A Conceptual Readiness Model}

Die Publikation von Muhammad Azeem Akbar, Saima Rafi und Arif Ali Khan (2022)~\cite{Akbar_2022} befasst sich mit der Migration von klassischer 
Software zu Quantensoftware und stellt ein konzeptionelles Readiness-Modell vor. Ziel der Arbeit ist es, ein Modell 
zu entwickeln, das Organisationen dabei unterstützt, ihre Fähigkeit zur Migration von klassischer zu Quantensoftware zu bewerten.

Das vorgeschlagene Modell basiert auf einer umfassenden Analyse bestehender Literatur, industrieller empirischer 
Studien und der Identifikation von Prozessbereichen, herausfordernden Faktoren und unterstützenden Elementen, die 
den Migrationsprozess beeinflussen. Es bietet einen strukturierten Ansatz zur Bewertung der Bereitschaft einer 
Organisation für die Migration zu Quantensoftware, indem es Best Practices und wichtige Prozessbereiche hervorhebt.

Ein zentrales Element des Modells ist die Berücksichtigung der spezifischen Herausforderungen, die mit der 
Entwicklung und dem Einsatz von Quantensoftware verbunden sind. Dazu gehören unter anderem die Komplexität der 
Quantentechnologie, die Notwendigkeit spezialisierter Fachkenntnisse und die Integration von Quanten- und 
klassischen Komponenten. Das Modell zielt darauf ab, diese Herausforderungen zu adressieren und Organisationen 
dabei zu unterstützen, geeignete Maßnahmen zur Vorbereitung auf die Migration zu Quantensoftware zu ergreifen.

Die Autoren betonen die Bedeutung eines strukturierten und systematischen Ansatzes für die Migration zu 
Quantensoftware, um die Effizienz und Effektivität des Migrationsprozesses zu maximieren. Sie argumentieren, dass 
das vorgeschlagene Modell als Leitfaden für Organisationen dienen kann, die sich auf den Übergang zu 
Quantensoftware vorbereiten, und dass es dazu beitragen kann, die Qualität und Nachhaltigkeit der 
migrierten Systeme zu gewährleisten.

Insgesamt bietet die Publikation wertvolle Einblicke in die Herausforderungen und Anforderungen der Migration 
von klassischer zu Quantensoftware und liefert ein praktisches Werkzeug zur Unterstützung dieses Prozesses.

\paragraph{Modelling Quantum Circuits with UML}

Die Publikation von Ricardo Pérez-Castillo, Luis Jiménez-Navajas und Mario Piattini (2021)~\cite{Perez-Castillo_2021} untersucht die Anwendung 
von Unified Modeling Language (UML) zur Modellierung von Quantenalgorithmen. Ziel der Arbeit ist es, eine 
UML-Erweiterung vorzustellen, die es ermöglicht, Quantenalgorithmen auf einer höheren Abstraktionsebene 
darzustellen und somit die Entwicklung und das Design von Quantensoftware zu erleichtern.

Die Autoren argumentieren, dass UML, eine weit verbreitete Modellierungssprache in der Softwareentwicklung, 
geeignete Erweiterungen benötigt, um die spezifischen Anforderungen der Quantencomputing-Entwicklung zu erfüllen. 
Sie stellen eine UML-Profile vor, das auf verschiedenen Stereotypen basiert und auf bestehenden UML-Aktivitätsdiagrammen 
angewendet werden kann, um Quantenoperationen und -gatter darzustellen. Diese Erweiterung ermöglicht es, 
Quantenalgorithmen zusammen mit klassischen Softwarekomponenten in integrierten Designs darzustellen.

Ein zentraler Vorteil dieser Methode ist die Möglichkeit, Quantenalgorithmen in einer standardisierten und 
weithin unterstützten Modellierungssprache darzustellen, was die Interoperabilität und Wiederverwendbarkeit 
der Modelle erhöht. Darüber hinaus ermöglicht die UML-Erweiterung die Integration von Quanten- und klassischen 
Softwareelementen in einem einheitlichen Modell, was die Zusammenarbeit und Kommunikation zwischen verschiedenen 
Entwicklungsteams erleichtert.

Die Publikation hebt auch die Herausforderungen und Potenziale der UML-Modellierung für Quantenalgorithmen hervor. 
Die Autoren diskutieren die Notwendigkeit, komplexe Quantenoperationen und -gatter präzise und verständlich zu 
modellieren, und betonen die Bedeutung von Modelltransformationen zur Generierung von Quantenprogrammen aus den UML-Modellen.

Insgesamt bietet die Arbeit wertvolle Einblicke in die Modellierung von Quantensoftware und zeigt auf, wie 
etablierte Modellierungstechniken aus der klassischen Softwareentwicklung auf das aufstrebende Feld des 
Quantencomputings angewendet werden können.