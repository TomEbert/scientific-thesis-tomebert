Diese Arbeit befasste sich mit der Konzeption eines Rahmens für ein Low-Code-Framework für Quantencomputing-Anwendungen.
Die Motivation für diese Untersuchung liegt in der inhärenten Komplexität der Quantenprogrammierung, die eine 
hohe Einstiegshürde für Entwickler 
darstellt, und der vielversprechenden Möglichkeit, diese Barrieren durch Low-Code-Ansätze zu senken.

Zu Beginn wurde ein umfassender Überblick über die theoretischen Grundlagen der Low-Code-Entwicklung, des Model-Driven 
Engineerings und des Quantencomputings gegeben. Dabei wurde erläutert, wie Low-Code-Entwicklungsumgebungen durch die 
Abstraktion technischer Komplexitäten und die Bereitstellung intuitiver, grafischer Entwicklungswerkzeuge die Zugänglichkeit 
zur Softwareentwicklung erhöhen können. Ebenso wurden die Potenziale und Herausforderungen von MDE in der Softwareentwicklung beschrieben.

Die systematische Literaturrecherche, die im Rahmen dieser Arbeit durchgeführt wurde, identifizierte zahlreiche Studien, 
die relevante Ansätze und Tools im Bereich Low-Code und Quantencomputing beleuchteten. Dabei wurden sowohl Gemeinsamkeiten 
als auch Unterschiede in den verschiedenen Ansätzen herausgearbeitet. Es zeigte sich, dass viele der bestehenden 
Low-Code-Entwicklungsplattformen bereits einige Aspekte der Quantenprogrammierung unterstützen, jedoch noch signifikante 
Forschungslücken bestehen. Im Rahmen der Arbeit wurde aufgezeigt wie ein prototypisches Low-Code-Framework für Quantencomputing 
konzipiert werden kann, das auf den Erkenntnissen der 
Literaturrecherche basiert. 

Die Analyse der identifizierten Publikationen zeigte mehrere Forschungslücken auf. Insbesondere besteht ein Bedarf an der 
Entwicklung robuster Low-Code-Tools, die eine breite Anwendbarkeit und Anpassungsfähigkeit bieten. Diese Anforderung 
gilt gleichermaßen für Open-Source-Lösungen wie auch für kommerzielle Produkte. Weiterhin ist es erforderlich, die 
Effizienz und Skalierbarkeit solcher Low-Code-Frameworks im Kontext des Quantencomputings zu verbessern.

Insgesamt zeigt diese Arbeit, dass die Integration von Low-Code-Entwicklungsansätzen in die Quantencomputing-Entwicklung zwar 
vielversprechend ist, jedoch noch in einem frühen Stadium steckt. Zukünftige Forschung sollte sich darauf konzentrieren, die 
identifizierten Lücken zu schließen und bestehende Frameworks gezielt zu optimieren. Ein konkreter Ansatzpunkt für die weitere 
Forschung wäre die prototypische Implementierung eines solchen Low-Code-Frameworks für Quantencomputing, das in bestehende 
Entwicklungsumgebungen und Pipelines integriert wird. Dabei sollten insbesondere die Transformation und Ausführung von 
Quantenalgorithmen berücksichtigt werden, um die Nutzungsmöglichkeiten für eine breitere Entwicklergemeinschaft zu erweitern. 