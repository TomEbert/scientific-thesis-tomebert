Diese Arbeit befasste sich mit der Konzeption und Entwicklung eines Low-Code-Frameworks für Quantencomputing-Anwendungen, 
unter besonderer Berücksichtigung von Model-Driven Engineering (MDE) und Open-Source-Prinzipien. Die Motivation für diese 
Untersuchung liegt in der inhärenten Komplexität der Quantenprogrammierung, die einen hohen Einstiegspunkt für Entwickler 
darstellt, und der vielversprechenden Möglichkeit, diese Barrieren durch Low-Code-Ansätze zu senken.

Zu Beginn wurde ein umfassender Überblick über die theoretischen Grundlagen der Low-Code-Entwicklung, des Model-Driven 
Engineerings und des Quantencomputings gegeben. Dabei wurde erläutert, wie Low-Code-Entwicklungsumgebungen durch die 
Abstraktion technischer Komplexitäten und die Bereitstellung intuitiver, grafischer Entwicklungswerkzeuge die Zugänglichkeit 
zur Softwareentwicklung erhöhen können. Ebenso wurden die Potenziale und Herausforderungen von MDE in der Softwareentwicklung beschrieben.

Die systematische Literaturrecherche, die im Rahmen dieser Arbeit durchgeführt wurde, identifizierte zahlreiche Studien, 
die relevante Ansätze und Tools im Bereich Low-Code und Quantencomputing beleuchteten. Dabei wurden sowohl Gemeinsamkeiten 
als auch Unterschiede in den verschiedenen Ansätzen herausgearbeitet. Es zeigte sich, dass viele der bestehenden 
Low-Code-Entwicklungsplattformen bereits einige Aspekte der Quantenprogrammierung unterstützen, jedoch noch signifikante 
Forschungslücken bestehen.

Im Rahmen der Arbeit wurde aufgezeigt wie ein prototypisches Low-Code-Framework für Quantencomputing konzipiert werden kann, das auf den Erkenntnissen der 
Literaturrecherche basiert. 

Die Analyse der identifizierten Publikationen offenbarte zudem mehrere Forschungslücken, darunter die Notwendigkeit, robuste 
Open-Source-Tools zu entwickeln, die eine breite Anwendbarkeit und Anpassungsfähigkeit bieten. Weitere Forschung ist auch 
erforderlich, um die Effizienz und Skalierbarkeit solcher Low-Code-Frameworks im Quantencomputing-Kontext zu verbessern.

Insgesamt zeigt diese Arbeit, dass die Integration von Low-Code-Entwicklungsansätzen in die Quantencomputing-Entwicklung 
vielversprechend ist, jedoch noch in den Kinderschuhen steckt. Zukünftige Forschung sollte sich darauf konzentrieren, die 
identifizierten Lücken zu schließen und die bestehenden Frameworks weiter zu optimieren. Die Ergebnisse dieser Arbeit 
liefern wichtige Impulse und eine solide Grundlage für die Weiterentwicklung von benutzerfreundlichen, effizienten und 
zugänglichen Entwicklungswerkzeugen im Bereich des Quantencomputings.